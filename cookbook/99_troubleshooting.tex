\chapter{Troubleshooting}

\section{PDT}

\begin{itemize}
  \item {\bf The PDT does not pass as the jar file is built with the wrong Java
  version.} Download the whole Peano project (from sourceforge via subversion),
  change into the directory \texttt{pdt}, and run
  \begin{code}
  ant createParser
  ant compile
  ant dist
  \end{code}
  Use the \texttt{pdt.jar} from the \texttt{pdt} directory now. 
\end{itemize}



\section{Programming}

\begin{itemize}
  \item {\bf How can I find out the location of a cell}. See the documentation
  of the \texttt{VertexEnumerator}. The object has routines to query cell
  position, size and level in the spacetree.
  \item {\bf I need the state object in my mapping}. My recommendation is
  clearly to create a copy of the state object in \texttt{beginIteration} and to
  merge this copy back in \texttt{endIteration}. The reason is simple: Peano
  updates the State itself. If you work with your own copy right from the start,
  you can be sure that you do not interfere with Peano's internal state at any
  time---even if you use shared memory parallelisation. There are however
  situations where you need the ``real'' state: If you plug into the MPI events
  and want to find out whether a rank just has been forked, e.g. In this case,
  you can store a pointer to the state in \texttt{beginIteration} and then
  access the state object. Keep in mind however that this version is not
  thread-safe.
\end{itemize}
