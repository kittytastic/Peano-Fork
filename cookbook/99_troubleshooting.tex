\chapter{Troubleshooting}

\section{PDT}

\begin{itemize}
  \item {\bf The PDT does not pass as the jar file is built with the wrong Java
  version}. Download the whole Peano project (from sourceforge via subversion),
  change into the directory \texttt{pdt/src}, and run
  \begin{code}
  make clean
  make createParser
  make compile
  make dist
  \end{code}
  Use the \texttt{pdt.jar} from the \texttt{pdt} directory now. 
\end{itemize}


%update-alterantives --config java



\section{Programming}

\begin{itemize}
  \item {\bf How can I find out the location of a cell}. See the documentation
  of the \texttt{VertexEnumerator}. The object has routines to query cell
  position, size and level in the spacetree.
  \item {\bf The adaptivity pattern lacks behing my adaptivity instructions}.
  For most refinement and erase calls, Peano needs at least one iteration to
  reliase them. It tries to do it faster, but usually needs these two
  iterations. Whenever Peano encounters regular subtrees, it tries to store 
  them separately in some arrays and process it way faster than with a classic
  tree. When a refinement command affects such a regular subregion, we first 
  have to find this region (at least one sweep), then break it up, in the
  follow-up iteration remove all veto markers that stopped the regular grid to
  be refined and then we can realise the refinement. If you have an extremely
  rapidely changing grid and you can't wait for Peano to keep pace, you should
  compile with \texttt{-DnoPersistentRegularSubtrees}. This will make the grid
  react quicker to your refinement requests but might slow down the code
  significantly.
  \item {\bf When does a grid coarsen?} If you invoke \texttt{erase}, Peano
  bookkeeps this coarsening requests but continues to traverse the grid. The
  actual grid then is actually coarsened in the subsequent grid sweep. If you
  trigger an erase, grid entities will not be destroyed prior to the next
  traversal. In some cases, it even can happen that an erase is neglected
  completely or postponed further:
  \begin{itemize}
    \item If your erase grid parts and refine adjacent grid parts, it can happen
    that the refine overwrites your erase (or the other way round). It depends
    which instruction can be accomodated by the grid first, i.e.~while Peano
    ensures that the grid remains valid and all events are always triggered in
    the right order, the actual grid pattern is non-deterministic from an
    application's point of view if you refine and coarsen at the same time in
    close domain regions.
    \item If you erase a grid region that is completely handled by one rank,
    then this makes the rank unemployed. If this happens, Peano first has to
    release the rank before the erase passes. In this case, you have to wait two
    or three iterations before your erase passes through. While the erase is
    bookkeeped but not realised yet, the state's \texttt{isGridStationary}
    attribute does not hold.
  \end{itemize}
  \item {\bf I need the state object in my mapping}. Create a copy of the state
  object in \texttt{beginIteration} and to merge this copy back in \texttt{endIteration}. 
  Peano updates the State itself and the state position in memory is not fixed.
  Do never ever hold a pointer to the state object handed into
  \texttt{beginIteration} or \texttt{endIteration}.
\end{itemize}


\section{Application tailoring}

\begin{itemize}
  \item {\bf The code requires too much memory}.
  You can try to compile your code with
  \linebreak \texttt{-DnoPersistentRegularSubtrees}.
  However, this may lead to a severe performance \linebreak
  degradation---notably if you run your code with shared memory parallelisation.
\end{itemize}

\section{MPI troubleshooting}

\begin{itemize}
  \item {\bf Does Peano support MPI-2?}. Yes, you can switch to MPI-2 if you add
  \texttt{-DMPI2}. I have experienced some issues with MPI-2 implementations and
  user-defined datatypes and thus decided to make MPI 1.3 the default. If you
  switch to MPI-2 and experience problems, you  mgiht want to have a look into
  any \texttt{records} directory and search for \texttt{initDatatype} to
  understand where issues arise.
  \item {\bf I've altered my data types and MPI starts to crash}. There are
  multiple reasons why user-defined data types start to fail. Here are some
  ideas to follow-up:
    \begin{enumerate}
      \item Compile with \texttt{-DAsserts}. We augment all parts of Peano with
      lots and lots of assertions, so they might help you to hunt down bugs.
      \item We have seen some compilers fail if you label some attributes of
      (vertex) data types with \texttt{expose}. Expose does alter the
      visibility, and we came to believe that these visibility modifications
      make some compilers reorder class attributes which in turns means that the
      MPI data types hold invalid byte offsets.
    \end{enumerate}
  \item {\bf My MPI version complains about a datatype \texttt{MPI\_CXX\_BOOL}}. 
    We have seen this with some older MPI versions. A quick fix is to translate
    your code with the additional argument
    \texttt{-DMPI\_CXX\_BOOL=MPI\_C\_BOOL}.
  \item To be continued \ldots
\end{itemize}


