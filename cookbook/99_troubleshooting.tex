\chapter{Troubleshooting}

% \section{PDT}
% 
% \begin{itemize}
%   \item \textbf{ The PDT does not pass as the jar file is built with the wrong
%   Java version}. Download the whole Peano project (from sourceforge via subversion),
%   change into the directory \texttt{pdt/src}, and run
%   \begin{code}
%   make clean
%   make createParser
%   make compile
%   make dist
%   \end{code}
%   Use the \texttt{pdt.jar} from the \texttt{pdt} directory now. 
% \end{itemize}


%update-alterantives --config java



\section{Compilation}

Peano relies on a header \texttt{tarch/compiler/CompilerSpecificSettings.h}.
Whenever I find incompatibilities between different compilers, I try to resolve them through this file. 
The header reads out some compiler preprocessor directives and then includes the
one it find most appropriate. 
You may always include your own file derived from one of the other headers in
the directory.


\begin{itemize}
  \item \textbf{ My compiler terminates with `error: unknown
   attribute 'optimize' ignored*`}. We have seen this issue notably on macOS
   with CLANG replacing GCC. Unfortunately, CLANG seems to pretend to be GNU on
   some systems and then the wrong header is included. Ensure that
   \texttt{CompilerSpecificSettings.h} includes a file such that the compiler
   \linebreak
   \texttt{CompilerCLANG} is defined.
  \item \textbf{ My compiler requires ages to translate the unit tests}. Ensure
  that the flag \linebreak \texttt{UseTestSpecificCompilerSettings} is enabled.
  This effectively disables any optimisation for the unit tests. Without the
  flag, unit test can become really big due to template instantiation.
  \item \textbf{configure crashes when I try to translate with TBB}. I have seen
  this on a couple of systems and found it annoying. First, ensure that configure
  find the TBB header. To ensure it does, check that \texttt{CXXFLAGS} are properly
		set. Next, set \texttt{LDFLAGS=-Lmypath}. Usually, I add \texttt{-ltbb} to 
  this parameter and then everything is fine. However, some configures then don't 
  like the configure script anymore. So I instead use \texttt{export PEANO_LDFLAGS="-ltbb"}. 
  For some systems, it then still remains necessary to add pthreads, too.
\end{itemize}



% \section{Debugging}
% \begin{itemize}
%   \item \textbf{gdb says ``not in executable format: file format not
%   recognized''}. automake puts the actual executable into a \texttt{.libs}
%   directory and creates bash scripts invoking those guys. Change into
%   \texttt{.libs} and run gdb directly on the executable. Before you do so,
%   ensure that \texttt{LD\_LIBRARY\_Path} points to the directory containing the
%   libraries. Again, those guys are stored in a \texttt{.libs} subdirectory, so
%   the library path should point to that subdirectory. Most of these tweaks
%   should not be necessary if you install Peano properly through \texttt{make
%   install}.
%   \item To be continued \dots
% \end{itemize}
% 
% 
% 
% 
% \section{Programming}
% 
% \begin{itemize}
%   \item \textbf{ How can I find out the location of a cell}. See the
%   documentation of the \texttt{VertexEnumerator}. The object has routines to query cell
%   position, size and level in the spacetree.
%   \item \textbf{ The adaptivity pattern lacks behing my adaptivity
%   instructions}.
%   For most refinement and erase calls, Peano needs at least one iteration to
%   reliase them. It tries to do it faster, but usually needs these two
%   iterations. Whenever Peano encounters regular subtrees, it tries to store 
%   them separately in some arrays and process it way faster than with a classic
%   tree. When a refinement command affects such a regular subregion, we first 
%   have to find this region (at least one sweep), then break it up, in the
%   follow-up iteration remove all veto markers that stopped the regular grid to
%   be refined and then we can realise the refinement. If you have an extremely
%   rapidely changing grid and you can't wait for Peano to keep pace, you should
%   compile with \texttt{-DnoPersistentRegularSubtrees}. This will make the grid
%   react quicker to your refinement requests but might slow down the code
%   significantly.
%   \item \textbf{ When does a grid coarsen?} If you invoke \texttt{erase}, Peano
%   bookkeeps this coarsening requests but continues to traverse the grid. The
%   actual grid then is actually coarsened in the subsequent grid sweep. If you
%   trigger an erase, grid entities will not be destroyed prior to the next
%   traversal. In some cases, it even can happen that an erase is neglected
%   completely or postponed further:
%   \begin{itemize}
%     \item If your erase grid parts and refine adjacent grid parts, it can happen
%     that the refine overwrites your erase (or the other way round). It depends
%     which instruction can be accomodated by the grid first, i.e.~while Peano
%     ensures that the grid remains valid and all events are always triggered in
%     the right order, the actual grid pattern is non-deterministic from an
%     application's point of view if you refine and coarsen at the same time in
%     close domain regions.
%     \item If you erase a grid region that is completely handled by one rank,
%     then this makes the rank unemployed. If this happens, Peano first has to
%     release the rank before the erase passes. In this case, you have to wait two
%     or three iterations before your erase passes through. While the erase is
%     bookkeeped but not realised yet, the state's \texttt{isGridStationary}
%     attribute does not hold.
%   \end{itemize}
%   \item \textbf{ I need the state object in my mapping}. Create a copy of the
%   state object in \texttt{beginIteration} and to merge this copy back in \texttt{endIteration}. 
%   Peano updates the State itself and the state position in memory is not fixed.
%   Do never ever hold a pointer to the state object handed into
%   \texttt{beginIteration} or \texttt{endIteration}.
%   \item \textbf{Peano's technical architecture uses a too precise machine
%   precision}. By default, we use $10^{-12}$ as machine precision. You can change
%   this however by compiling with \linebreak
%   \texttt{-DMACHINE\_PRECISION=0.000001}, e.g.
% \end{itemize}
% 
% 
% \section{Application tailoring}
% 
% \begin{itemize}
%   \item \textbf{ The code requires too much memory}.
%   You can try to compile your code with
%   \linebreak \texttt{-DnoPersistentRegularSubtrees}.
%   However, this may lead to a severe performance \linebreak
%   degradation---notably if you run your code with shared memory parallelisation.
% \end{itemize}
% 
% \section{TBB troubleshooting}
% 
% \begin{itemize}
%   \item \textbf{ I get tons of warning alike ``warning \#603: ``typeid'' is reserved for future use as a keyword'' }.
%   Peano's pdt adds a compile flag \texttt{-fno-rtti} to the makefile as the core framework does 
%   not need type information at runtime and codes thus become smaller and faster. TBB however 
%   seems to use runtime information, so you have to remove this flag from your makefile if you 
%   want to get rid of all the warning. 
% \end{itemize}
% 
% \section{MPI troubleshooting}
% 
% \begin{itemize}
%   \item \textbf{ Does Peano support MPI-2?}. Yes, you can switch to MPI-2 if you
%   add \texttt{-DMPI2}. I have experienced some issues with MPI-2 implementations and
%   user-defined datatypes and thus decided to make MPI 1.3 the default. If you
%   switch to MPI-2 and experience problems, you  mgiht want to have a look into
%   any \texttt{records} directory and search for \texttt{initDatatype} to
%   understand where issues arise.
%   \item \textbf{ I've altered my data types and MPI starts to crash}. There are
%   multiple reasons why user-defined data types start to fail. Here are some
%   ideas to follow-up:
%     \begin{enumerate}
%       \item Compile with \texttt{-DAsserts}. We augment all parts of Peano with
%       lots and lots of assertions, so they might help you to hunt down bugs.
%       \item We have seen some compilers fail if you label some attributes of
%       (vertex) data types with \texttt{expose}. Expose does alter the
%       visibility, and we came to believe that these visibility modifications
%       make some compilers reorder class attributes which in turns means that the
%       MPI data types hold invalid byte offsets.
%     \end{enumerate}
%   \item \textbf{ My MPI version complains about a datatype
%   \texttt{MPI\_CXX\_BOOL}}.
%     We have seen this with some older MPI versions. A quick fix is to translate
%     your code with the additional argument
%     \texttt{-DMPI\_CXX\_BOOL=MPI\_C\_BOOL}.
%   \item \textbf{ My code deadlocks once I increase the rank count. It seems that
%     the ranks cannot even register at the global node pool anymore.}
%     I've seen this bug with Intel MPI and Omnipath recently. It seems that some
%     MPI implementations struggle to handle many polling non-blocking operations:
%     If multiple ranks are deployed on one node, $k$ of them register at the node
%     pool and then eagerly poll the global master for work. As a result, the
%     $k+1$th rank cannot send out its registration message. The global master in
%     return does not start the actual computation before all ranks have
%     registered if you use the corresponding wait call. There are multiple
%     solutions/things you can try:
%     \begin{enumerate}
%       \item Remove the \texttt{waitForAllNodesToBecomeIdle} calls from your
%       code. Your code might not need it anyway.
%       \item Deploy only one MPI rank per node/interconnect.
%       \item Change into the directory \texttt{tarch/compiler} and find the right
%       compiler-specific header for your system. Change Peano's load balancing
%       data exchange into a blocking MPI:
%       \begin{code}
%       #define SendAndReceiveLoadBalancingMessagesBlocking    -1
%       \end{code}
%     \end{enumerate}
%     However, the best solution is to consult your supercomputer's documentation
%     and to configure the fabric accordingly. On Durham's supercomputer, e.g., an
%     additional
%     \begin{code}
%     export I_MPI_FABRICS="tmi"
%     \end{code} 
%     in the SLURM script fixes the issue.
%   \item \textbf{ Once I switch to (Intel) MPI, my code seems to receive invalid
%     data}. We have seen this problem with Intel's MPI (but not with OpenMPI, 
%     e.g.) and it seems that this one makes very strong assumptions about the
%     alignment of data within arrays to optimise the MPI message exchange. I
%     recommend that you write some ping pong tests (one rank sending stuff to
%     another and then this one sending stuff back where the data is compared to
%     the original) and that you ping pong both single vertices and arrays of
%     vertices. This typically uncovers data inconsistencies. One next step then
%     is to translate your code with \texttt{-DnoPackedRecords}. While switching
%     the flag on reduces the memory footprint dramatically (if you don't use
%     lots of double arrays that is), its underlying skipping of the compiler's
%     padding seems to cause problems for Intel MPI. If you find 
%     \texttt{-DnoPackedRecords} working, roll back, i.e.~use 
%     \texttt{-DPackedRecords} (which is the default), and follow Section
%     \ref{section:advanced-mpi:tailor-data-exchange-format}.
%   \item To be continued \ldots
% \end{itemize}



\section{External tools}
\begin{itemize}
  \item \textbf{The Intel tools yield invalid or messed up results}. Ensure 
    you have built your code with \texttt{-DTBB\_USE\_ASSERT -DTBB\_USE\_THREADING\_TOOLS}.
  \item To be continued \dots
\end{itemize}



