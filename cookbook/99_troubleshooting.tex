\chapter{Troubleshooting}

\section{PDT}

\begin{itemize}
  \item {\bf The PDT does not pass as the jar file is built with the wrong Java
  version}. Download the whole Peano project (from sourceforge via subversion),
  change into the directory \texttt{pdt}, and run
  \begin{code}
  ant createParser
  ant compile
  ant dist
  \end{code}
  Use the \texttt{pdt.jar} from the \texttt{pdt} directory now. 
  \item {\bf I can't use the ant script to recreate the pdt.jar as it complains
  with ``Error: Could not find or load main class
  org.apache.tools.ant.launch.Launcher''}. I did run into this problem on a
  couple of recently new Linux systems (mainly OpenSUSE). All the workarounds
  from the Internet failed on my system. What works (surprisingly) is to install
  Eclipse which internally supports ant and right click on the
  \texttt{build.xml} file and select \texttt{Run As Ant Build \ldots}
\end{itemize}


%update-alterantives --config java



\section{Programming}

\begin{itemize}
  \item {\bf How can I find out the location of a cell}. See the documentation
  of the \texttt{VertexEnumerator}. The object has routines to query cell
  position, size and level in the spacetree.
  \item {\bf I need the state object in my mapping}. Create a copy of the state
  object in \texttt{beginIteration} and to merge this copy back in \texttt{endIteration}. 
  Peano updates the State itself and the state position in memory is not fixed.
  Do never ever hold a pointer to the state object handed into
  \texttt{beginIteration} or \texttt{endIteration}.
\end{itemize}


\section{Application tailoring}

\begin{itemize}
  \item {\bf The code requires too much memory}.
  You can try to compile your code with \texttt{-DnoPersistentRegularSubtrees}.
  However, this may lead to a severe performance degradation---notably if you
  run your code with shared memory parallelisation.
\end{itemize}
