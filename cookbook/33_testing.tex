\chapter{Unit tests}
\label{section:test}


Peano introduces its own lightweight test concept with picks up best practices
from the JUnit world. 
While it is simple and has no external dependencies (all test routines are
collected within the \texttt{tarch} subdirectory), it provides a few macros to
handle numeric data---which might be convenient for AMR applications; so
there's some added value. It also provides test routines for vectors.


\section{Test architecture}

A Peano test is a class that implements and extends
\texttt{tarch::tests::TestCase}.
By convention, I place my tests always into subdirectories/subnamespaces of the
respective component, but this is not a must.
A test class must implement a \texttt{void run()} method.
This is the only convention. 
The Peano toolkit by default generates one (empty) test per project that can act
as a blueprint if you don't want to write your test manually. 
Feel free to delete the class if you don't want to write unit tests on your own
(though the test itself is empty and thus does not harm the test runs).

A proper test class should realise a couple of \texttt{void} operations without
arguments that run the individual tests.
A minimalistic test thus looks as follows:

\begin{code}
namespace myproject {
  namespace tests {
    class TestCase;
  }
}

class myproject::tests::TestCase : public tarch::tests::TestCase {
 private:
  /**
   * These operations implement the real tests.
   */
  void test1();
  void test2();
 public:
  TestCase();
  virtual ~TestCase();

  virtual void run();
};
\end{code}

\noindent
The test's \texttt{run} operation has to register the individual void
  routines running the actual tests:
  \begin{code}
void exahype::tests::TestCase::run() {
  testMethod(test1);
  testMethod(test2);
}
  \end{code}




\section{Running the tests}
Once these steps are handled, you can run the tests within your \texttt{main}.
For the core and tarch routines, I provide factory methods  

\begin{code}
tarch::getUnitTests();
peano4::getUnitTests();
\end{code}

\noindent
which deliver a test suite. To run them, I typically insert snippets alike
\begin{code}
  int unitTestsErrors = 0;
  tarch::tests::TestCase* tests = nullptr;

  tests = tarch::getUnitTests();
  tests->run();
  unitTestsErrors += tests->getNumberOfErrors();
  delete tests;
\end{code}

\noindent
into my main.
It returns an output similar to
\begin{code}
running test case collection "exahype2.tests.c" ...... ok                                                                                                                
running test case collection "exahype2.tests.solvers" .. ok                                                                                                              
running test case collection "exahype2.tests" .... ok                                                                                                                    
running test case collection "peano4.grid.tests" .................... ok                                                                                                 
running test case collection "peano4.grid"  ok                                                                                                                           
running test case collection "peano4.heap.tests" .. ok                                                                                                                   
running test case collection "peano4.heap"  ok                                                                                                                           
running test case collection "peano4"  ok                                                                                                                                
running test case collection "tarch"  ok                                                                                                                                
running global test case collection  ok                                                                                                                                 
\end{code}

\noindent
The output returns a hierarchical overview over all tests ran. 
The number of dots per package identifies the number of actual test functions in
a namespace.
If tests fail, the interface reports the exact location including file name and
line number.


\begin{remark}
  It is convenient to run tests only if you have translated your code with a 
  sufficient debug level (cmp.~Section \ref{section:logging}).
  When you write your own tests, I often add test attributes to the
  classes---fields that are not required but help me to validate object
  states. Since I protect these attributes with ifdefs reading Peano's debug
  level, they are removed from the production code.
\end{remark}



\section{Test routines}

A typical unit tests builds up some classes, befills them with data, invokes
some operations and finally checks the class state via some getters or routines
returning results.
Peano's way to do the actual tests are \texttt{validate} macros.
You find all these macros in the file \texttt{tarch/tests/TestMacros.h}.
There are three classes of macros:

\begin{itemize}
  \item Standard validation macros similar to assertions that test a boolean
  condition.
  \item Comparison macros that compare two values through the
  \texttt{operator==} (which might be overloaded).
  \item Comparison macros that compare doubles, vectors or matrices numerically,
  i.e.~up to a prescribed precision.
\end{itemize}

\noindent
All of these macros come along with variants that allow you to plot additional
information if a test has failed, i.e.~to plot the values of further variables,
e.g.
All validation macros automatically set a test to failed if they fail, and they
write a verbose message on the reason why some tests failed to the terminal.  



  Tests tend to become spaghetti code and some of them are very expensive to
  translate as the compiler's optimiser has a hard time to translate the long
  code fragments.
  I thus recommend to insert
  \begin{code}
#include "tarch/compiler/CompilerSpecificSettings.h"

#ifdef UseTestSpecificCompilerSettings
#pragma optimize("", off)
#endif

// All the implementation

#ifdef UseTestSpecificCompilerSettings
#pragma optimize("", on)
#endif
  \end{code}
  into your implementation file around all your code. This switches off all
  optimisation explicitly for the tests.


