\chapter{Quickstart}

\section{Download and install}

To start work with Peano, you need at least two things.

\begin{enumerate}
  \item The Peano source code. Today, the source code consists of two important
  directories. The \texttt{peano} directory holds the actual Peano code. An
  additional \texttt{tarch} directory holds Peano's technical architecture.
  \item The Peano Development Toolkit (PDT). The PDT is a small Java archive. It
  takes away the cumbersome work to write lots of glue code, i.e.~empty
  interface implementations, default routines, \ldots, so we use it quite
  frequently.
\end{enumerate}

\noindent
For advanced features, you might want to use some {\bf toolboxes}.
A toolbox in Peano is a small collection of files that you store in a directory
and adopt all pathes accordingly.
From a user's point of view, when we use the term toolbox we actually mean this
directory with all its content.



\begin{history}
Originally, we hoped that Peano's technical architecture (\texttt{tarch}) might
become of value for several projects, i.e.~project appreciate that they do not
have to re-develop things such as logging, writing of output files, writing
support for OpenMP and TBB, and so forth.
To the best of our knowledge, the tarch however is not really used by someone
else, so we cannot really claim that it is independent of Peano.
Nevertheless, we try to keep it separate and not to add anyting AMR or
grid-speicific to the tarch.
\end{history}

There are two ways to get hold of Peano's sources and tools. You either {\em
download the archives from the website} or you {\em access the repository
directly}.
Both variants are fine.
We recommend to access the respository directly.


\subsection{Download the archives from the website}

If you don't want to download Peano's whole archive, simply to Peano's webpage
\url{http://www.peano-framework.org} and grab the files
\begin{itemize}
  \item \texttt{peano.tar.gz} and
  \item \texttt{pdt.jar}
\end{itemize}
from there. If you do so, please skip the first two lines from the script
before. Otherwise, load down the important files with \texttt{wget}. 
Independent of which variant you follow, please unpack the \texttt{peano.tar.gz}
archive. 
It holds all required C++ sources.

\begin{code}
> wget http://sourceforge.net/projects/peano/files/peano.tar.gz
> wget http://sourceforge.net/projects/peano/files/pdt.jar
> tar -xzvf peano.tar.gz
\end{code}


\noindent
There's a couple of helper files that we use IN the
cookbook. 
They are not necessarily required for each Peano project, but for our examples
here they are very useful.
So, please create an additional directory \texttt{usrtemplates} and grap
these files

\begin{code}
> mkdir usrtemplates
> cd usrtemplates
> wget http://sourceforge.net/projects/peano/files/ \
  usrtemplates/VTKMultilevelGridVisualiserImplementation.template 
> wget http://sourceforge.net/projects/peano/files/ \
  usrtemplates/VTKMultilevelGridVisualiserHeader.template 
> wget http://sourceforge.net/projects/peano/files/ \
  usrtemplates/VTKGridVisualiserImplementation.template 
> wget http://sourceforge.net/projects/peano/files/ \
  usrtemplates/VTKGridVisualiserHeader.template 
> wget http://sourceforge.net/projects/peano/files/ \
  usrtemplates/VTK2dTreeVisualiserImplementation.template 
> wget http://sourceforge.net/projects/peano/files/ \
  usrtemplates/VTK2dTreeVisualiserHeader.template 
\end{code}

\subsection{Access the repository directly}

Instead of a manual download, you might also decide to download a copy of the
whole Peano repository. 
This also has the advantage that you can do a simple \texttt{svn update} anytime
later throughout your development to immediately obtain all kernel
modifications.


\begin{code}
> svn checkout http://svn.code.sf.net/p/peano/code/trunk peano
\end{code}

\noindent
Your directory structure will be slightly different than in the example above,
but this way you can be sure you grabbed everything that has been released for
Peano through the webpage ever.

The archive \texttt{pdt.jar} will be contained in \texttt{pdt}, while the two
source folders will be held by \texttt{src}.
The directory \texttt{usrtemplates} is contained in \texttt{pdt}.


\subsection{Prepare your own project}


From hereon, we recommend that you do not make any changes within Peano
repositories but use your own directory for your particular project.
We call this project from hereon \texttt{myproject}.
Within \texttt{myproject}, we will need to access the directories \texttt{peano}
and \texttt{tarch}.
It is most convenient to create symbolic links to these files.
Alternatively, you also might want to copy files around or adopt makefiles,
scripts, and so forth.
I'm too lazy to do so and rely on OS links.


\begin{code}
> mkdir myproject
> cd myproject
> ln -s ../../mydownloads/peano peano
> ln -s ../../mydownloads/tarch tarch
> ls
  peano   tarch
\end{code}
