\section{Peano's block-structured output format}


Peano has introduced its own block-structured output format.
Its idea is that a mesh is dumped as an unstructured set of cell where each cell
is uniquely identified through its offset and its size. 
The mesh data format lacks any topological information (connectivity).
A cell always is a patch, i.e.~holds a (topological) Cartesian mesh of
dimensions $k \times k$ or $k \times k \times k $, respectively.
If you work without patches, $k=2$. 
In this case, the block format obviously can yield a significant overhead.



\subsection{A minimal ASCII example file}
\begin{code}
#
# Peano output file
# Version 0.1
#
format ascii
dimensions 3
patch-size 6 6 6

begin vertex-values "identifier A"
  number-of-unknowns 58
  meta-data "This is some fancy meta data"  
end vertex-values

begin vertex-values "time"
  number-of-unknowns 1
  meta-data "This is some fancy meta data"  
end vertex-values

begin patch
  offset 0.0 0.0 0.0
  size   0.5 0.5 0.5
  begin vertex-values "identifier A"
    18.0 123.0 ...
  end vertex-values
  begin vertex-values "time"
    1.0 1.0 ...
  end vertex-values
end patch 

begin patch
  offset 0.5 0.0 0.0
  size   0.25 0.25 0.25
  ...
end patch 
\end{code}

\noindent
This spec file specifies a simple grid that

\begin{itemize}
  \item The actual grid plotted consists of two patches (boxes discretised with
  a small regular grid).
  \item Both small grids consist of $6 \times 6 \times 6$ cells. They all have
  the same size, i.e.~are equally spaced. 
  \item Each vertex holds two sets of unknowns. The first set is called
  \texttt{identifier A} and each entry (per vertex) comprises 58 double
  unknowns. The second set of unknowns per vertex is called \texttt{time} and is
  a scalar quantity.
  \exahype\ always writes some timestamp data.
  If you do not use local/anarchic time stepping, all time stamps
  typically are the same.
  \item The data per cube is ordered lexicographically, i.e.~we run along the
  x-axis first, then along the y-axis, and finally along the z-axis. 
  \item All data is held as SoA, i.e.~we first give all the 58 unknowns of the
  first vertex within the cube, then the 58 unknowns of the right neighbour
  vertex, and so forth.
  \item All output data is ASCII.
  \item Patches may overlap in both space and time.
\end{itemize}


\noindent
If you use Peano's/\exahype's file format, please note that you typically obtain
a bunch of files. 
A meta file whose name coincides with the one you specify in your specification
file. 
This meta file contains links to the actual data files. 
Its syntax is self-explaining and you can open it with a plain text editor.


\subsection{More sophiciated dumps}

\begin{itemize}
  \item The file format supports \texttt{cell-values}.
%   \item In our example, we stick to ASCII data. The file format however also
%   supports \texttt{binary} data in which case everything within
%   \texttt{vertex-values} and \texttt{cell-values} is dumped as binary stream.
%   Finally, we plan to provide \texttt{hdf5} data dumps in which case only one
%   integer is stored within the \texttt{vertex-values} or \texttt{cell-values}
%   sections which identifies the row in the hdf5 data dump.
%   \item Other data file fragments can be included via \texttt{include}
%   statements.
  \item Each \texttt{vertex-values} or \texttt{cell-values} section in the
  header of the file, i.e.~not embedded into a patch, may have a section
  \begin{code}
   begin mapping
     0.0 0.0 0.0
     0.1 0.1 0.1
     0.4 0.4 0.4
     ...
   end mapping
  \end{code} 
  If no mapping is present, our code dumps a regular subgrid (patch) per
  \texttt{patch} region. If a mapping is present, the mapping has exactly $(6+1)
  \cdot (6+1) \cdot (6+1)$ entries in the example from the previous section.
  Each entry is a 3d coordinate relative to the unit cube and specifies how the
  topologially regular grid prescribed within a patch is to be mapped. In
  \exahype, mappings are used to store the Gau\ss -Legendre spacing.
  \item The plotter by default always creates two types of files: The actual
  data files as described above and one meta file. The meta file solely links to
  the actual data:
  \begin{code}
# 
# Peano patch file 
# Version 0.1 
# 
format ASCII
begin dataset
  include "conserved-1-rank-0.peano-patch-file"
end dataset
begin dataset
  include "conserved-2-rank-0.peano-patch-file"
end dataset
  \end{code}
  Per snapshot (typically time step), the plotter adds on \texttt{dataset}
  entry. Each data set holds one data file per active rank. Each rank writes its
  actual data into a separate file. 
\end{itemize}


\subsection{Peano's HDF5 format}

The lack of connectivity data renders the streaming into HDF5 tables
straightforward despite the fact that we have to be able to handle dynamically
adaptive meshes.
The HDF5 files are described by a hierarchy (Figure \ref{xxxx}):

\begin{itemize}
  \item The root entry of the file holds an attribute \texttt{Number of
  datasets} which is sole integer value.
  \item For a given number of datasets, there are groups \texttt{dataset-x} with
  \texttt{x} starting from 0.
\end{itemize}
