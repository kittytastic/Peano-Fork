\section{Performance analysis}
\label{section:performance-analysis}

\chapterDescription
  {
    Less than 10 minutes unless you postprocess a big file.
  }
  {
    You may work with the plain output that Peano writes to the terminal. If you
    use log filters (cmp.~Chapter \ref{section:logging}), it is important that
    you know how to switch particular logging infos on. You also need a working
    Python installation.
  }

\noindent
Prior to any parallelisation or tuning discussion, I want to emphasise that it
usually makes sense first of all to have an idea how Peano is performing from a
grid point of view. For this, the framework comes along with a rather useful script.
From hereon, I recommend to use some state-of-the-art performance analysis tool
to translate the application point of view into a more technical context.

\subsection*{Peano's built-in performance analysis tools}

\begin{itemize}
  \item Recompile your code with \texttt{-DPerformanceAnalysis}. For the
    performance analysis, it usually makes sense to compile with the highest
    optimisation level and to disable \texttt{-DDebug} and \texttt{-DAsserts}.
  \item Ensure that  you use the \texttt{CommandLineLogger} as output device.
    This is the default Peano logger, so unless you have written your own log
    format, no further action should be necessary.
  \item Ensure that your \texttt{CommandLineLogger} plots the time stamp in a
    machine-readable way, that it uses tabs or spaces as entry separators, it
    plots the trace, and it plots the machine name (if you use MPI). There's an operation on the logger that allows you to do
    exactly this:
    \begin{code}
tarch::logging::CommandLineLogger::getInstance().setLogFormat(
  " ",    // columnSeparator has to be a space or a tab
  true,   // logTimeStamp has to be set to true
  ...,    // logTimeStampHumanReadable is not important for the performance analysis
  true,   // logMachineName has to be true if you profile MPI codes
  ...,    // logMessageType is not relevant for the performance analysis 
  true,   // logTrace has to be set
  ...
);
    \end{code} 
  \item Run your code and ensure that \texttt{info} outputs from the
    \texttt{peano::performanceanalysis} component are enabled.
  \item Pipe the output into a file:
    \begin{code}
> ./myExecutable myArguments > outputfile.txt
    \end{code} 
    We call this file \texttt{outputfile.txt} from hereon.
  \item Pass the output file to Peano's performance analysis script written in
  Python. 
    \begin{code}
> python <mypath>/peano/performanceanalysis/performanceanalysis.py 
    \end{code} 
    Calling it without any options displays a usage message.
  \item Open the web browser of your choice and the generated files (typically
  your Peano output file with an additional \texttt{html} extension).
\end{itemize}

\begin{remark}
The present document suggests to pipe all output data from the terminal into a
file and to process this file. 
For many applications with low rank count this works fine.
The more ranks you have however the higher the probability that concurrent
writes to the terminal mess up your piped file.
In most cases, the performance analysis still succeeds. 
However, there are cases where the postprocessing fails.
In this case, it is better to use \texttt{setLogFormat} of the
\texttt{CommandLineLogger} to make the logger pipe output from different ranks
into different files.
There's an additional script in the \texttt{performanceanalysis} directory that
you can then use to fuse the various output files into one file before you start
the actual postprocessing.
\end{remark}



\noindent
If you browse through your directory, you will notice that all graphs are
written both as png and as pdf. 
You can thus integrate them directly into your \LaTeX\ reports.
Most generated graphs also are available in an increased resolution. 
These files have a \texttt{.large.} inserted into their extension.


\subsection*{Study only particular phases of your code}

If you run the whole performance analysis over a longer Peano simultion, the
amount of data produced quickly becomes massive and hard to handle, and the generation of
these massive data even might pollute your results. 
In your code, you can always switch all performance analysis off. 
To do so, invoke
\begin{code}
  peano::performanceanalysis::Analysis::getInstance().enable(false);
\end{code}

\noindent
and call the same routine later on again with a \texttt{true} argument.

In a parallel environment, you have to switch the analysis on or off per rank,
i.e.~the analysis does not coordinate via MPI. 

\subsection*{Score-P/Scalasca}

We use Peano with Score-P\footnote{\url{http://www.vi-hps.org/projects/score-p}}
which is used by Scalasca\footnote{\url{http://http://www.scalasca.org/}}, e.g.
Score-P however tends to yield massive amounts of data. 
Peano can manually turn Score-P data aquisition on/off within the code. 
To enable this feature,
\begin{enumerate}
  \item retranslate your code with \texttt{-DUseScoreP},
  \item ensure you use the performance analysis as detail ed above, and
  \item switch the performance analysis on and off suiting your demands as it is
  detailed in the previous paragraphs.
\end{enumerate}


\noindent
Internally, Peano's performance analysis deactivates and activates the Score-P
tracing through
\begin{code}
SCOREP_RECORDING_ON()
SCOREP_RECORDING_OFF()
\end{code}

\noindent 
Furthermore, Peano annotates its routines/phases with traces through the 
\begin{code} 
SCOREP_USER_REGION("...", SCOREP_USER_REGION_TYPE_FUNCTION)
\end{code}
command added as preamble to its core routines. 
The string is the trace of the routine.


To translate an instrumented\footnote{
  We have faces issues with installing Score-P on our OpenSUSE installations which went away when I changed into 
  \texttt{/lib64} and created a link there: \texttt{ln -s libz.so.1 libz.so}.
} 
code, please use
\begin{code}
scorep --noonline-access --nocompiler --mpp=mpi  --thread=none --user mpicxx
scorep --noonline-access --nocompiler --mpp=none  --thread=none --user icpc
\end{code}


\subsection*{Intel's VTune Amplifier}

Intel's VTune Amplifier sometimes yields data that is very hard to interpret if
you use TBB, our preferred multithreading code base. 
It is convenient to translate the code with \linebreak
\texttt{-DTBB\_USE\_THREADING\_TOOLS} and thus to enrich the executable (also
ensure \texttt{-g} is enabled such that the profiler can map insight onto code
fragments.


TBB's work stealing traditionally results in enormous spin times. To clarify
where these guys are coming from, you should recompile with \texttt{-debug all}.
However, this effectively switches off the optimisation. It is therefore only to
be used to dive into performance issues you have identified with your optimised
code.


