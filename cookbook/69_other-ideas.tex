\section{Other ideas}

This part has yet to be written, but I need documentation on 

\begin{itemize}
  \item \texttt{-DnoParallelExchangePackedRecordsAtBoundary}
  \item \texttt{-DnoParallelExchangePackedRecordsBetweenMasterAndWorker}
  \item \texttt{-DnoParallelExchangePackedRecordsInHeaps}
  \item \texttt{-DnoParallelExchangePackedRecordsThroughoutJoinsAndForks}
\end{itemize}

%

%Become topology-aware


% Throughout the bottom-up traversal, each mpi traversal first receives data from 
% all its children, i.e. data deployed to remote traversals, and afterward sends 
% data to its master in turn. Unfortunately, Peano has to do quite some 
% algorithmic work after the last children record has been received if and only 
% if some subtrees are also to be traversed locally. It hence might make sense to
% introduce pure administrative ranks that do not take over any computation on the finest grid level. 
% Again, we do a brief 1d toy case study:
% 
% foobar
% 
% In the upper case, the blue rank triggers the red one to traverse its subtree. The red one in turn tiggers 3 and 4. Afterward, it continues with 2 and then waits for 3 and 4 to finish. After the records from 3 and 4 have been received, it has to send its data to 0 to allow 0 to terminate the global traversal. However, between the last receive and the send, some administrative work has to be done, as the red node also holds local work (it has to run through the embedding cells to get the ordering of the boundary data exchange right, but that's irrelevant from a user point of view). This way, we've introduced an algorithmic latency: Some time elaps between 3 and 4 sending their data and the red one continuing with the data flow up the tree. This latency becomes severe for deep Splittings.

