\chapter{Performance Analaysis}
\label{capter:performance-analysis}


\Peano's support for performance analysis tools rests on two legs:
On the one hand, the code has its built-in statistics tracking which samples
statistics from a user perspective.
On the other hand, we provide support for a couple of tracing APIs.
The classes underlying both features all are located within
\texttt{tarch::logging}.
The statistics is only one class therein.
All the tracing relies on \Peano's logging interface.


\section{Statistics}

The collection of statistics is switched on by compiling with
\texttt{-DTrackStatistics}.
To configure the statistics, consult the class documentation of
\texttt{tarch::logging::Statistics}.
\Peano\ collects user-defined counters and dumps them all as a CSV file if you
invoke the dump operation on the statistics. 
You have to do so automatically\footnote{\ExaHyPE\ automatically trigger this
operation for you.}.



\section{Tracing}


All tracing relies on \Peano's internal tracing API.
That is, each instrumented function invokes the tracing functions.
The tracing then filters (does the user want to see these things or not).
``Successfull'' traces then are forwarded into the tracing backend and written
in the right format.
Section \ref{section:logging} discusses how to use the tracing.


Here's the steps you have to do:
\begin{enumerate}
  \item Set \texttt{-DUsedLogService=device} as compile flag to select the
  correct tracing target. All loggers are described in Section
  \ref{section:logging:logging-devices}. I recommend to rerun \texttt{configure}
  to propagate the compiler flag all the way through.
  \item Recompile the whole code/\Peano\ core.
  \item Link against the \texttt{\_trace} libraries of \Peano\footnote{In
  \ExaHyPE, there is a corresponding target that you can select.}.
  \item Ensure your own code is compiled with \texttt{-DPeanoDebug=1} (or
  higher). See the discussion on page \pageref{section:logging}.
  \item Ensure that you have a log filter file and that those routines that you
  are interested in are whitelisted.
\end{enumerate}





