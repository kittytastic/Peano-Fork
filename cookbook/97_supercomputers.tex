\chapter{Selected HPC platforms}


\section{Supercomputer login and change into working directory}

\paragraph{SuperMUC-NG}
\begin{code}
 ssh lu57zat6@skx.supermuc.lrz.de
 cd $WORK
\end{code}


\paragraph{Hamilton}
\begin{code}
 ssh frmh84@hamilton.dur.ac.uk
 module load intel/xe_2018.2
 module load intelmpi/intel/2018.2
 cd $SCRATCH
 cd /ddn/data/frmh84/
 setenv CXX icpc
\end{code}

From hereon, the steps are all the same. If you wanna use TBB, you however have to set

\begin{code}
 setenv CXXFLAGS $TBB_INC
 setenv LDFLAGS "$TBB_SHLIB"

 ./configure --with-multithreading=tbb --with-mpi=mpicxx
\end{code}

before you call configure from above. Be sure you use the hyphens above. The TBB library flag comes
along with two arguments and you want to move both over to \texttt{LDFLAGS}.



\section{Quick install}

Please exchange the login below with your login. 
Login is only possible from machines with a fixed, registered IP. 
Durham's mira server is one example for such a machine. 



\subsection{Grab from raw git files (not recommended)}

\begin{code}
 git clone https://gitlab.lrz.de/hpcsoftware/Peano.git
 cd Peano
 git checkout p4
 libtoolize; aclocal; autoconf; autoheader; cp src/config.h.in .; automake --add-missing
\end{code}

\section{Configure and build}

\begin{code}
 export CXX=icpc
 ./configure --with-multithreading=cpp --with-mpi=mpicxx
 --prefix=$(WORK)/Peano/bin make -j
\end{code}


% So geht es auch auf dem Login-Knoten:
% export I_MPI_HYDRA_BOOTSTRAP=fork


