\chapter{Selected HPC platforms}


\section{SuperMUC-NG}

%
Some remarks on Peano4 on SuperMUC-NG:


\begin{code}
 ssh lu57zat6@skx.supermuc.lrz.de
 cd $WORK
 export CXX icpc
\end{code}



% So geht es auch auf dem Login-Knoten:
% export I_MPI_HYDRA_BOOTSTRAP=fork

\section{Hamilton}

Some remarks on Peano4 on Hamilton, Durham's local supercomputer:

\begin{code}
 ssh frmh84@hamilton.dur.ac.uk
 module purge
 # module load intel/xe_2018.2
 module load intel/2019.5
 module load intelmpi/intel/2019.6
 # for the Python API
 module load python/3.6.8 
 module unload gcc/8.2.0
 # for legacy DaStGen
 module load java/1.8.0
 cd $SCRATCH
 cd /ddn/data/frmh84/
 setenv CXX icpc
 ./configure --with-multithreading=what-you-prefer --with-mpi=mpicxx
\end{code}

\noindent
It is not clear to me, why the Python module does load gcc. 
In my opinion, it should not, i.e.~the unload command should not be required.
Anyway, if you skip the unload, then the Intel linker will fail.

The Java module is required if and only if you use the legacy DaStGen feature to
model data structures.

If you you Intel MPI, you either have to use the \texttt{-parallel} flag
when you compile with icpc---otherwise, your linker will fail---or you
use the \texttt{mpicxx} wrapper right from the start.


\begin{code}
 setenv CXX mpicxx
\end{code} 

\noindent
It is clear that your Slurm scripts have to load the same environment. 

\paragraph{Intel's Threading Building Blocks (TBB)}

Intel's Threading Building Blocks are available through the environment
variables \texttt{TBB\_INC} and \texttt{TBB\_SHLIB}, as soon as you have 
loaded the Intel compiler.
To pipe them into configure, we have to map them onto the appropriate autotools
parameters:

\begin{code}
 setenv TBB_INC "-I/ddn/apps/Cluster-Apps/intel/xe_2018.2/tbb/include"
 setenv TBB_SHLIB "-L/ddn/apps/Cluster-Apps/intel/xe_2018.2/tbb/lib/intel64/gcc4.7 -ltbb"
 
 setenv TBB_INC "-I/ddn/apps/Cluster-Apps/intel/2019.5/tbb/include"
 setenv TBB_SHLIB "-L/ddn/apps/Cluster-Apps/intel/2019.5/tbb/lib/intel64/gcc4.7 -ltbb"
 
 setenv CXXFLAGS "$TBB_INC  -DTBB_USE_ASSERT -DTBB_USE_THREADING_TOOLS"
 setenv LDFLAGS "$TBB_SHLIB -ltbb_debug"
 setenv CXX icpc
 ./configure --with-multithreading=tbb --with-mpi=mpicxx
\end{code}



\section{Local workstations}
If you don't have a proper module file, you might have to configure your environment manually to use TBB:
\begin{code}
export CXXFLAGS=-I/opt/intel/tbb/include
export LDFLAGS="-L/opt/intel/tbb/lib/intel64/gcc4.7 -ltbb -pthread"

./configure --with-multithreading=tbb --with-mpi=/opt/mpi/mpicxx
\end{code}

I found bash sometimes to be picky when setting the TBB libraries. The path and the library literally have to be set separately.






