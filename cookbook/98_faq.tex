\chapter{FAQs (programmers' point of view)}



% \section{General}
% 
% 
% \begin{itemize}
%   \item \textbf{Is there any profiling/debugging support within \Peano?} I'm
%   planning to support ITAC tags/macros one day in the core, but at the moment
%   there's no particular debugging support. You have to be sure you link against
%   the release libraries of Peano (use the \texttt{Release} mode in your Python
%   script if you use Python and select which library to link against), and you
%   might want to switch to the
%   Google Chrome output (Chapter \ref{section:logging:logging-devices}). The
%   latter gives you some nice timelines. Please note that the release version
%   still does dump some data (cmp.~Chapter
%   \ref{chapter:installation:build-variants}) which you have to disable manually.
%   \item To be continued \dots
% \end{itemize}
% 
% 
% 
% \section{\ExaHyPE}
% 
% 
% \begin{itemize}
%   \item To be continued \dots
% \end{itemize}



\section{C++ core}

\begin{itemize}
  \item \textbf{Where is the information whether a point or face is at the
  boundary or not?}
  I do not provide any such feature. Many codes for example use the grid to
  represent complicated domains, and thus would need such a feature anyway. So
  what you have to do is to add a bool to each vertex/face and set this boolean
  yourself.
  \item To be continued \dots
\end{itemize}


\section{Parallelisation}

\begin{itemize}
  \item \textbf{Why do I (temporarily) get an adaptive grid even though I
  specify a regular one?}
  This phenomenon arises if your adaptive refinements and (dynamic) load
  balancing happen at the same time. Load balancing is realised by transfering a
  whole tree part (subpartition incl.~coarser scales) to another rank after a
  grid sweep. Refinement happens in multiple stages: After a grid traversal, the
  rank takes all the refinement instructions and then realises them throughout
  the subsequent grid sweep. So if a rank gets a refinement command and then
  gives away parts of its mesh, then it might not be able to realise this 
  refinement. At the same time, the rank accepting the new partition is not
  aware of the refinement requests yet. So it might get the refinement
  request, but with one step delay: It sets up the local partition, evaluates
  the refinement criterion, is informed about a refinement wish for this
  area, and subsequently realises it. For most codes, that delay by one
  iteration is not a problem as the newly established rank will just realise the
  refinement one grid sweep later, but the point is:
  refine and erase commands in \Peano\ are always a wishlist to the kernel. The
  kernel can decide to ignore it---at least for one grid sweep.
  \item To be continued \dots
\end{itemize}
