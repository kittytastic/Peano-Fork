\section{Matrix-free multigrid}
  \label{section:applications:matrix-free-multigrid}

\chapterDescription
  {
    1--2 days.
  }
  {
    Chapter \ref{chapter:quickstart}.
  }

In this section, we sketch how to solve the convection-diffusion equation
\[
  - \nabla (\epsilon \nabla) u + \nabla (v\ u) = f \qquad \mbox{with } v \in
  \mathbb{R}^d, \epsilon \in \mathbb{R}^{d \times d}
\]
with various geometric multigrid solvers. $epsilon$ is a diagonal matrix with
entries $\epsilon _1, \epsilon _2$ or $\epsilon _1, \epsilon _2, \epsilon _3$,
respectively.
Our realisation is based upon a few design decisions:

\begin{enumerate}
  \item The solvers use the spacetree as computational grid.
  \item We use a finite element formalism with $d$-linear shape functions.
  \item The material parameters $\epsilon $ and $v$ are given per cell.
\end{enumerate}


\noindent
For the implementation, we use Peano's \texttt{matrixfree} toolbox. 
This is a tiny little collection of helper classes to work with stencils and
local assembly matrix. 
It is neither fast, i.e.~computationally mature, nor can it cope with real
stencil libraries, but it does the job.



\subsection{Setup}

We start with Peano's PDT and generate a project. We also link Peano's sources
into the project and unzip the \texttt{matrixfree} toolbox. 
\begin{code}
  java -jar pdt.jar --create-project multigrid multigrid
  ln -s mypath/src/peano 
  ln -s mypath/src/tarch
  cp mypath/tarballs/toolboxes/matrixfree.tar.gz .
  tar -xzvf matrixfree.tar.gz
\end{code}


\noindent
We add our material parameters to the \texttt{Cell.def} file
\begin{code}
Packed-Type: short int;

Constant: DIMENSIONS;

class multigrid::records::Cell {  
  persistent parallelise double   epsilon[DIMENSIONS];
  persistent parallelise double   v[DIMENSIONS];
  persistent parallelise double   f;
};
\end{code}

\noindent
and create a simple first specification file:
\begin{code}
component: Multigrid

namespace: ::multigrid

vertex:
  dastgen-file: Vertex.def
  
cell:
  dastgen-file: Cell.def

state:
  dastgen-file: State.def

event-mapping:
  name: SetupGrid

event-mapping:
  name: PlotCells

adapter:
  name: CreateGrid
  merge-with-user-defined-mapping: SetupGrid
  merge-with-user-defined-mapping: PlotCells
\end{code}

\noindent
We run this specification file through the PDT

\begin{code}
java -jar pdt.jar --generate-gluecode multigrid/project.peano-specification multigrid
\end{code}


\noindent
and implement both the plotter and the creational mapping such that we have a
few characteristic setups. 

We next introduce an operation 
\begin{code}
matrixfree::stencil::ElementWiseAssemblyMatrix multigrid::Cell::getElementsAssemblyMatrix(
  const tarch::la::Vector<DIMENSIONS,double>&  h
) const {
  matrixfree::stencil::ElementWiseAssemblyMatrix result;

  const matrixfree::stencil::Stencil laplacianStencil = 
    matrixfree::stencil::StencilFactory::getLaplacian(_cellData.getEpsilon(), h);

  return matrixfree::stencil::ElementMatrix::getElementWiseAssemblyMatrix(laplacianStencil);
}
\end{code}
which returns the $\mathbf{R}^{2^d \times 2^d}$ local system matrix. 
The method sets up the stencils that correspond to a regular Cartesian system
given the given mesh size \textt{h} and the material parameters.
Here, also the convective term has to be handled.
Finally, it uses \texttt{getElementWiseAssemblyMatrix} to extract the actual
matrix from this stencil.

\begin{remark}
  Peano supports all stencil/la operations also for complex-valued equations.
\end{remark}


\subsection*{Further reading}

\begin{itemize}
  \item  Reps, Bram and Weinzierl, Tobias: t.b.d.
  \item Weinzierl, Marion: {\em Diss} t.b.d.
  \item   Muntean, Ioan Lucian, Mehl, Miriam, Neckel, Tobias and Weinzierl,
  Tobias (2008). {\em Concepts for Efficient Flow Solvers Based on Adaptive
  Cartesian Grids}. In High Performance Computing in Science and Engineering, Garching 2007. Wagner, Siegfried, Steinmetz, Matthias, Bode, Arndt & Brehm, Matthias Berlin Heidelberg New York: Springer.
  \item Weinzierl, Tobias and K\"oppl, Tobias (2012). {\em A Geometric
  Space-time Multigrid Algorithm for the Heat Equation}. Numerical Mathematics:
  Theory, Methods and Applications 5(1): 110-130.
  \item Mehl, Miriam, Weinzierl, Tobias and Zenger, Christoph (2006). {\em A
  cache-oblivious self-adaptive full multigrid method}. Numerical Linear Algebra
  with Applications 13(2-3): 275-291.
\end{itemize}
