\chapter{Logging, tracing and assertions}
\label{section:logging}


\section{Logging, tracing and assertion architecture}

Peano is built upon a technical architecture (\texttt{tarch}) which provides
logging, tracing and assertions.
The two directory of relevance here is therefore \texttt{tarch/logging} plus the
file \texttt{Assertions.h}.
The latter file holds all the assertion macros, while the logging directory
holds a file \texttt{Log.h} which holds all of the essential logging and tracing
macros.


When we build Peano, we do distinguish different ``debug'' levels:
\begin{itemize}
  \item If a code is translated with \texttt{-DPeanoDebug=0}, all logging,
  tracing and assertion macros are disabled, i.e.~replaced by empty statements.
  Warning, information and error messages are still displayed.
  \item If a code is translated with \texttt{-DPeanoDebug=1}, all macros from
  \texttt{-DPeanoDebug=0} remain defined. This time however, all tracing
  functionality is enabled. With this mode, it is possible to do production
  runs plus to extract some performance data, e.g.
  \item If a code is translated with \texttt{-DPeanoDebug=2}, all the
  tracing macros from \texttt{-DPeanoDebug=1} remain defined. On top of this,
  Peano's assertions now are enabled, i.e.~assertion macros do not degenerate to
  nop but actually run tests. This might increase the runtime significantly yet
  the code permanently runs test.
  \item If a code is translated with \texttt{-DPeanoDebug=4}, logging and
  assertion macros from \texttt{-DPeanoDebug=4} remain defined. Furthermore, any
  debug information written through Peano's log interface is now not discarded
  (removed at compile time) but actually shown. This is the slowest mode
  yielding most of the data.
\end{itemize}



\section{Using the logging and tracing interface}

Peano relies on a plain interface to write out user information. 
All constituents of this interface are collected in the package
\texttt{tarch::logging}.
Documentation on this package can be found in the corresponding header files,
files ending with \texttt{.doxys}, or the Peano webpage (section on sources).
The usage paradigm is simple:

\begin{enumerate}
  \item Each class that wants to write logging information requires an instance
  of class \texttt{tarch::logging::Log}. As such an instance is required per
  class, it makes sense to make this field a static one.
  \begin{code}
  #include "tarch/logging/Log.h"
  ...
  class MyClass {
    private:
      static tarch::logging::Log  _log;
  };
  
  tarch::logging::Log  MyClass::_log( "MyClass" );
  \end{code}
%   For most auto-generated classes, the PDT already creates the \texttt{\_log}
%   instance. 
  Please keep to the nomenclature of the class field to make all
  macros work. Please use as string argument in the constructor the fully
  qualified class name. See the remarks on log filters below for an explanation
  for  this.
  \item Whenever you want to log, trace, write an error message or a warning,
  you should use Log's operations to write the messages. Alternatively, you may want to use the log macros from
  \texttt{Log.h}. They work with stringstreams internally, i.e.~you may write
  things along the lines
  \begin{code}
logInfo( "runAsMaster(...)", "time step " << i << ": dt=" << dt );
  \end{code}
  where you concatenate the stream with data.
\end{enumerate}

Peano offers four levels of logging through its macros:
\begin{itemize}
  \item {\bf Info}. Should be used to inform your user about the application's
  state.
  \item {\bf Warning}. Should be used to inform your user about bothering
  behaviour.
  The MPI code uses it, e.g., if many messages arrive in a different order than
  expected. Messages written to the warning level are piped to
  \texttt{cerr}.
  \item {\bf Error}. Should be used for errors. Is piped to \texttt{cerr} as
  well.
  \item {\bf Tracing}. Used to keep track of when a method is entered and
  then left again.
  \item {\bf Debug}. Should be used to write debug data. It goes to
  \texttt{cout} and all debug data is removed if you do not translate with the compile flag
  \texttt{-DDebug}. Notably, use the \texttt{logDebug} macros when you write to
  the debug level, as all required log operations then are removed by the
  compiler once you create a release version of your code.
\end{itemize}




\section{Logging devices (formats)}
\label{section:logging:logging-devices}

The \texttt{Log} instance forwards the information to a logger. By default, this
is the \\ \texttt{tarch::logging::ChromeTraceFileLogger}. You may want to write
your own alternative implementation or switch to another option.
Logging can be very time-consuming. I therefore decided to link the Log objects
to one particular log device statically. If you want to ``redirect'' your
outcome, you can either manually edit the logging header, or you can retranslate
your code with \texttt{-DUseLogService=xxx}.
\texttt{xxx} is the class to be used.
I detail the variants I currently ship with Peano below.






\subsection{CommandLineLogger}

To select this logger, compile with
\texttt{-DUseLogService=CommandLineLogger}.
This logger is the default.


This logger dumps all information in a human readable format.
It can be tailored towards your needs.
For this, the logger provides a particular setter.
Please consult the header or the webpage for details on the semantics of the
arguments:
\begin{code}
tarch::logging::CommandLineLogger::getInstance().setLogFormat(
  " ", true, false, false, true, true, "my-fancy.log-file" );
\end{code} 

\noindent
The interface also allows you to pipe the output into a file rather than to the
terminal.
This is particular useful for MPI applications, as each rank is assigned a file
of its own and messages are not screwed up.
Typically, the logger is configured in the \texttt{main} of the application.

If you run Peano for long simulations and, notably, if you run Peano with debug
information switched on, log files soon become massive.
To ease the pain, Peano's command line logger offers an operation 
\begin{code}
tarch::logging::CommandLineLogger::getInstance().closeOutputStreamAndReopenNewOne();
\end{code} 
\noindent
that works if you have specified an output file before (see
\texttt{setLogFormat} above). Per close, you close all output files and start to
stream the output into a new file.
Typically, developers use this operation in their iterative schemes to stream
each iteration to a different file.
The output files are enumerated automatically.



\subsection{ChromeTraceFileLogger}

To select this logger, compile with
\texttt{-DUseLogService=ChromeTraceFileLogger}.

This logger writes the one json file per rank. The format conforms to Google
Chrome's tracing format, i.e.~you can open the file in Google Chrome by typing
in the URL \texttt{chrome:///tracing} and loading in the file in the GUI that
pops up.
As the format is prescribed, there are no opportunities to configure the output.
While the majority of all output data goes into these Chrome files, warnings,
errors and info messages are also dumped to the terminal in a format which can
be read by humans easily.



\subsection{NVTX logger (NVIDIA SDK)}

To select this logger, compile with
\texttt{-DUseLogService=NVTXLogger}.

This logger forwards all trace commands to NVIDIA's profiling/tracing library. 


\subsection{ITAC logger}

To select this logger, compile with
\texttt{-DUseLogService=ITACLogger}.

This logger forwards all trace commands to Intel's profiling/tracing library. 



\section{Log filter}
\label{section:logging:log-filter}

The amount of log information often becomes hard to track; notably if you run in
debug mode.
Often, you are interested only in a subset of all log messages.
For this, Peano offers log filters which provide a blacklist and whitelist
mechanism to filter messages before they are written.
A log filter entry is created by 
\begin{code}
tarch::logging::LogFilter::getInstance().addFilterListEntry( tarch::logging::LogFilter::FilterListEntry(
  tarch::logging::LogFilter::FilterListEntry::TargetInfo, 
  tarch::logging::LogFilter::FilterListEntry::AnyRank, "peano4", false,
  tarch::logging::LogFilter::FilterListEntry::AlwaysOn
));
\end{code} 


\noindent
The log filter is a singleton, and log filters can be used totally independently
of the log output format.


Configuring log filters in your source code is a convenient option when you
start a new project.
On the long run, it is cumbersome if you have to recompile every time you want
different log information.
Therefore, the \texttt{CommandLineLogger} also offers a routine that allows you
to load log filter entries from a text file.
This facilitates work with log filters.
The usage is straightforward
\begin{code}
tarch::logging::LogFilterFileReader::parsePlainTextFile( "my.log-filter" );
\end{code}

\noindent
and the format is very simple:
{\footnotesize
\begin{verbatim}
# Level             Trace                    Rank          Black/white  When on
# info/debug/trace  trace of class affected  -1=all ranks
  debug             tarch                    -1            black       always-on
  debug             peano4                   -1            black       always-on
  info              tarch                    -1            black       always-on
  info              peano4                   -1            black       always-on
  trace             tarch                    -1            black       always-on
  trace             peano                    -1            black       always-on

# Lines starting with hash are comments

# Switch on info output from one class on one rank only
info     examples::subdirectory::myClass    15    white  my-phase
\end{verbatim}
}

\noindent
For performance reasons, it is important to take two facts into account:
\begin{enumerate}
  \item Black- and whitelisting works only on the class level. While tracing,
  logging, and so forth all tell you which operation has logged a certain entry,
  you cannot filter on the method level.
  \item Black- and whitelisting are not dynamic. The very first time a class
  wants to dump information, it queries the log filter. The result of the log
  filter then is hold persistently. 
\end{enumerate}


\begin{remark}
 Whether a log message is written or not results from the combination of compile
 flags (\texttt{PeanoDebug} flag) plus the log filters you have in place. If you 
 translate with \texttt{-DPeanoDebug=0}, debug log filters play no role, as the 
 underlying messages are not written anyway.
 In this context, you have to pay attention against which libraries you link. 
 Peano builds the technical architecture, the core and tools in different
 versions with different optimisation levels and different log settings. 
 Select the right one if you wanna see messages from the core or tarch.
\end{remark}


\noindent
You can have different log entries for different program phases.
By default, the log filter does not keep track of the program phase, i.e.~it
does not know which program phase is on.
You can use 
\begin{code}
tarch::logging::LogFilter::getInstance().switchProgramPhase("my-phase");
\end{code}

\noindent
From hereon, only the log filter entries labelled with \texttt{my-phase} or with
\texttt{always-on} will be used.
You can use any string for \texttt{my-phase}.
Some higher-level frameworks such as \ExaHyPE\ define some program phases a
priori.

\section{Assertions}

Peano provides its own assertion format which is independent of C++ assertions.
Assertions are offered through macros, and if you choose a \texttt{PeanoDebug}
level which does not support assertions, then they are removed at compile time
from your code.

Peano's assertions interlink with the logging, i.e.~before they shut down a
rank, they ensure that all logs are flushed.
Furthermore, they provide a few convenient variations that 
\begin{enumerate}
  \item support numerical ``equality'' checks,
  \item support the logging of parameters, and
  \item support vector-valued comparisons.
\end{enumerate}
