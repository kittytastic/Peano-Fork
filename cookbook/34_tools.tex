\chapter{Tools}
\label{section:tools}


\section{d-dimensional loops}

I wrote Peano to support arbitrary spatial dimensions.
By default, I build only 2d and 3d versions of the code/core, but in principle
you can write also simulations over 4d or 5d domains, e.g.
The choice of a particular dimension d is done at compile time via the
flag \texttt{-DDimension=d}.
As everything is generic w.r.t.~dimensions, I have many places in the code where
I need d-dimensional loops.


Peano's utility directory/namespace \texttt{peano4::utils} offers a header
\texttt{Loop} which comprises a set of macros that support d-dimensional loops.
Details can be found in the source code documentatation within the file or
online, but the collection of macros allows you to write code alike

\begin{code}
  dfor(k,20) { // d-dimensional loop over 20x20x20x... patch
    // There is now a d-dimensional vector k over integers which holds the 
    // loop index. So we can write for example
    const tarch::la::Vector<Dimensions,double> somePosition = 
      offset + k.convertScalar<double>() * h;
    [...]
  }
\end{code}

\noindent
There are also macros for loops skipping one dimension, or macros for particular
iteration ranges (2,3,4) which are highly optimised through lookup tables.
Finally, \texttt{Loop} hosts z-loops, i.e.~loops which meander forth and back
through an array similar to the Peano curve.



