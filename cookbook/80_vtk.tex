\chapter{The visualisation toolkit (VTK) and integrated IO}
\label{chapter:vtk}

\section{Installing \Peano\ with VTK support}
Installing VTK support can be challenging. 
Here's some things you might want to consider/check:

\begin{itemize}
  \item To build the \Peano\ conversion/visualisation tools, you have to
  translate the code with \texttt{--with-vtk}. If you don't
  specify a path, then \Peano\ assumes that all VTK is installed 
  within \texttt{/usr/include}\footnote{Recent Ubuntu
  versions for example tend to place VTK in subdirectories of
  \texttt{/usr/local/include}, while SUSE et al indeed place them directly
  within \texttt{/usr/include}. In any case, you will have to select a
  subdirectory from these folders, as VTK usually goes into separate subfolder (depending on their
  version). If you have not installed VTK as stand-alone but as part of the
  Paraview developer version, e.g., then you have to search for the VTK
  directory within the Paraview folder. In any case, search for a
  filte \texttt{vtkVersion.h}. It is the folder hosting this file that we are
  searching for here.}.
  \item We next have to know which VTK version you are using. VTK builds a
  different set of libraries in each generation. \Peano\ has to know which
  version you want to use to link against the right set of libraries. The
  installation script will automatically try to detect your version, but the
  detection is not very robust. If it is the wrong version, use the
  \texttt{--with-vtk-version=x} switch to set the version manually. We currently
  support \texttt{x} 7,8,9, i.e.~we are only interested in the major version of
  your VTK installation. If you have version 8.90, it is the 8 we are interested
  in.
  \item Next, the script will search for the right libraries. By default, our
  script assumes that the libraries are have a suffix with the major dot minor
  number. So we assume that the VTK version 8.90 yields a library
  \texttt{vtkIOCore-8.90}. This is the default. The script will try to find this
  default library and give you some feedback (be careful: if it doesn't find
  the library, it will still continue as you might want to change the pathes
  later on). If your library naming convention is different---we've seen
  systems dropping the version numbers or Paraview installations which append
  something alike \texttt{pv8.90}---then specify your suffix manually through
  \texttt{--with-vtk-suffix}. If your installation does not have a version suffix,
  as is the case in Fedora, you should pass an empty string to this:
  \texttt{--with-vtk-suffix=''}
  \item If your compile passes through but fails in the linking state (object
  not found), then you have to check if the VTK libraries are in the search
  path. If they are not (very likely), then add them prior to the compile call:
\begin{code}
export LDFLAGS="-L/opt/vtk/lib64"
\end{code}
\end{itemize}


%   \item If you configure with \texttt{--with-vtk}, but the makefile cannot find
%   the headers, you have to add the correct search path\footnote.
%   You can either set the
% \begin{code}
% export CPLUS_INCLUDE_PATH=$CPLUS_INCLUDE_PATH:/usr/include/vtk
% \end{code}
%   or you set the corresponding environment variable prior to the configure
%   script. I always recommend that latter route, as our Python API inherits all of the configure
%   variables, i.e.~it will then know where to search for VTK:
% \begin{code}
% export CXXFLAGS=$CXXFLAGS:"-I/usr/local/include/vtk-8.90"
% \end{code}
%   The first setting btw stems from a Fedora system, the latter example from an
%   Ubuntu installation, while OpenSUSE in our test farm got the includes
%   automatically right.


%  Depending on the version chosen, \Peano\ will try to link against a different
%  set of library files.
    
% -with-vtk-version=


  
  
  
  
%   Many VTK installations append the version number to the library, i.e.~they create a library \texttt{libvtkCommonCore-8.90.so} instead of
%   \texttt{libvtkCommonCore.so} if you use VTK 8.9. 
%   \Peano tries to detect this version number automatically but in case it fails,
%   you can use the options \texttt{--with-vtk=<path>} and \texttt{--with-vtk-version=<MAJOR.MINORS>}
%   to specify exactly the Peano version you want to use.
%   If you are not sure, run \texttt{find / -name libvtk*.so} and see what it finds. 
%   %After that, tell the
%   %configuration explicitly which VTK extension to use:
%   %\begin{code}
%   %--with-vtk-version=8.90
%   %\end{code}
%   BTW: Not all Linux distributions install VTK with these extensions, so you
%   have to check manually on your system.

    
\section{Tweaking your pictures}

Peano's default VTK plotters project Peano's tree-based grid onto an
unstructured mesh, as VTK does not support tree meshes.
The underlying mechanism is straightforward:

\begin{enumerate}
  \item If a vertex is touched/read the first time, it is dumped as vertex of
  the unstructured output mesh if it is not refined, i.e.~if no other vertex
  does exist at the same location on a finer mesh level.
  \item If a hanging vertex is created, we add a new vertex to the output dump
  if this hanging vertex has not been written before. The plotters typically
  maintain one big hash map to bookkeep which hanging vertices have already been
  written.
  \item If the tree traversal runs into an unrefined cell, this cell is made a
  cell of the unstructured output mesh.
\end{enumerate}


The block format/the conversion tools in contrast generate all the points
multiple times.
If you have to cells sharing a vertex, the vertex is dumped twice.
For patch-based problems, this is okish, as the number of vertices that are
shared is way smaller than the total number of vertices in the domain (vertices
within a patch are not dumped multiple times).
As a consequence of the redundant vertex dumps, isosurfaces, e.g., might not
work.
If you need them, it is reasonable to project/map the actual data onto a regular
mesh first.


If your PDE has $m$ quantities, Peano's block format dumps them into a vector of
cardinality $m$ per vertex or cell, respectively. 
Most visualisation filters/tools however require data to be either scalar or 
vector data. 
In Paraview, use the calculator filter to extract only some quantities from the 
data dump.
Alternatively, you can ask the convert tool to extract only some data into the
vtu output.

\noindent
Please note that most visualisation codes (such as Paraview) do interpolate
bi-/tri-linearly within the cells. 
If a cell's face is adjacent to cells on finer levels, hanging vertices on the
face exist.
If they do not hold the linearly interpolated value, you will discover
visualisation artifacts.

\begin{remark}
  For smooth output pictures of meshes with adaptivity, you have to 
  \begin{itemize}
    \item set the plotted properties on hanging vertices due to $d$-linear
    interpolation, and
    \item you have to add the plotter mapping after you have invoked the mapping
    that does initialise the hanging vertices.
  \end{itemize}
\end{remark}


\noindent
Please note that the dumped unstructured mesh is a non-conformal mesh. 
Algorithms such as isosurface identification thus might yield invalid
results---it depends on your visualisation algorithms.


% \section{Using \Peano's Python front-end to interactively visualise data}
% 
% I've tested a couple of Python options how to visualise 3d data produced by 
% \Peano\ directly from the Python API. 
% None of them really did appeal to me or had support for Python 3 (cmp.~Mayavi2).
% The only think that seems to work is to use Paraview as a server, and to connect from your
% Python \Peano\ terminal to this server.
% \Peano's Python API offers some tools to do so.
% 
% 
% \begin{remark}
%  If you build Paraview yourself, you have to enable its Python support. If you
%  use a prepackaged Paraview, you will have to ensure that it includes the
%  developer and Python parts. For Ubuntu, e.g., we need three packages in total:
%  \texttt{paraview}, \texttt{paraview-python} and \texttt{paraview-dev}.
% \end{remark}
% 
% The tools wrap around recommendations/examples from
% \url{https://www.paraview.org/Wiki/ParaView/Python_Scripting}.
% Before launching any
% \Peano\ script, fire up your Paraview server
% \begin{code}
% > pvserver
% Waiting for client...
% Connection URL: cs://YourMachine:11111
% Accepting connection(s): YourMachine:11111
% \end{code}
% 
% \noindent
% and take note of the connection details, i.e.~URL. 
% 
% 
% \begin{code}
% 
% \end{code}
% 
% 
% \url{https://www.paraview.org/Wiki/ParaView_and_Python}
% 
% 
% #
% # Test
% #
% 
% from paraview.simple import *
% 
% reader = vtkXMLUnstructuredDataWriter( "solutionParallelEuler-0-ParallelEulerQ-domain-decomposition-fine-grid.vtu")
% 
% #cone = Cone()
% #cone.Resolution = 32
% #cone.Center = [1, 2, 3]
% GetActiveSource()
% Shrink()
% SetActiveSource(cone)
% shrinkFilter = Shrink(cone)
% shrinkFilter.UpdatePipeline()
% Show(shrinkFilter)
% Render()
% input()
% 
% 
% 
% 
% #
% # Der Teil funktioniert noch net so richtig ;-(
% # ---------------------------------------------
% 
% #servermanager.Connect( "Bad" )
% #print( str(GetActiveSource()) )
% #cone = Cone()
% #print( str(GetActiveSource()) )
% #SetActiveSource(cone)  # Braucht man wohl net
% #
% #print( str(GetActiveSource()) )
% #
% #Show()
% #Render()
% 
% #cone = Cone()
% #cone.Resolution = 32
% #cone.Center = [1, 2, 3]
% #GetActiveSource()
% #Shrink()
% #SetActiveSource(cone)
% 
% #shrinkFilter = Shrink(cone)
% #shrinkFilter.UpdatePipeline()
% #Show(shrinkFilter)
% #Render()
% #input()
% 
% #Delete(cone)
% 
% 
% #
% # Es gibt ne Datei servermanager.py, die Bespiele enthaelt. Die geht bei mir aber auch net.
% #
% #connection  = servermanager.Connect( "Bad", 11111 )
% #filters.AppendDatasets()
% #render_view = CreateRenderView()#
% 
% #input()
% 
