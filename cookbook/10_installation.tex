\chapter{Installation}

There are two ways to obtain \Peano: 
You can either download one of the archives we provide on the webpage, or you
can work directly against a repository clone.
If you work with the archive, type in a 
\begin{code}
tar -xzvf myarchive.tar.gz
\end{code}
in the directory where you've stored your downloaded file.


If you work with the git archive, you have to clone this archive first. 
I grant access to \Peano\  free of charge.
However, I ask users to sign up for the software if they indend to push
modifications to the code (which I very much appreciate\footnote{\Peano's
guidebook (the file you currently read) is hosted within \Peano's git
repository, too. I'm always happy if people add content to this
documentation, too.}).
This way I can report to funding agencies how frequent the software is used, and
I also have at least some ideas which application areas benefit from the
software and where it is actively used and developed.
If you do not indend to modify the \Peano\ core code base, you can just clone
the code anomymously.

\begin{code}
git clone https://gitlab.lrz.de/your-login/Peano.git
cd Peano
git checkout p4
\end{code}


\begin{remark}
I still maintain the ``old'' Peano in the repository (version 3), and most users
consider this to be the standard Peano generation.
For the present document, it is thus important that you manually switch to the
branch \texttt{p4}.
\end{remark}



\section{Prepare the configure environment}

This step should only required if you work directly against the git repository.
If you prefer to download a snapshot of \Peano, then you can skip this section
and continue with \ref{section:installation:configure}.


\begin{itemize}
  \item Ensure you have the autotool packages installed on your system. They
  typically are shipped in the packages \texttt{autoconf}, \texttt{automake} and
  \texttt{libtool}.
  \item Set up the configure environment: 
 \begin{code}
 libtoolize; aclocal; autoconf; autoheader; 
 cp src/config.h.in .; 
 automake --add-missing
 \end{code}
\end{itemize}


\noindent
These steps should only be required once, unless you push major revisions to the
development branch.


\section{Configure}
\label{section:installation:configure}


\Peano\  relies on the autotools to set up its build environment.
Change into the project's directory and type in 
\begin{code}
./configure --help
\end{code}


The \texttt{--help} option should give you some information about the available
variants of \Peano.
In principle, a sole \texttt{./configure} call is sufficient, but you might want
to adopt your build; notably as the default build is serial and does not bring
along any support for postprocessing.
While the help output should be reasonable verbose, I summarise key options
below:

\begin{center}
 \begin{tabular}{lp{10cm}}
  \texttt{--prefix=/mypath} & Will install \Peano\  in \texttt{/mypath}.
   \\
  \texttt{--with-multithreading} & Switch on multithreading. By default, we
  build without multithreading, but a \texttt{--with-multithreading=cpp}, e.g.,
  makes \Peano\  use the C++ threading model. Please consult \texttt{--help} for
  details.
   \\
  \texttt{--with-mpi} & Enable the MPI version of \Peano. You have to tell the
  build environment however which compile command to use. Please note that
  we need a C++ MPI wrapper. So \texttt{--with-mpi=mpcxx} is a typical call. 
   \\
  \texttt{--with-vtk} & Inform \Peano\  that VTK is available on the system and
  that it should build all the visualisation and postprocessing tools that rely
  upon VTK. The VTK installation is sometimes not easy (and you might have to
  provide additional parameters depending on your installation). I dedicate
  Chapter \ref{chapter:vtk} on VTK. 
   \\
  \texttt{--with-hdf5} & Make \Peano\  support HDF5 output. Not stable at the
  moment.
   \\
  \texttt{--with-delta} & Configure \Peano\  such that the geometry library
  $\Delta $ is used. Not stable at the moment.
 \end{tabular}
\end{center}


\begin{remark}
 I recommend that you start a first test without any additional flavours of
 \Peano, i.e.~to work with a plain \texttt{./configure} call. Once the tests
 pass, I recommend that you first add IO (VTK) and then parallelisation.
\end{remark}

\section{Build}

Once the configuration has been successful, a simple 
\begin{code}
make
\end{code}
should build the \Peano\ core and some examples.


\begin{code}
make install
\end{code}
finally will deploy the \Peano\ files in the directory specified via the
\texttt{prefix} before.
If you haven't set the prefix, then the default will likely be a system
directory.
Unless you have superuser rights, the installation then will fail.
So I recommend that you install \Peano\ into a local directory specifying it via
the \texttt{--with-prefix} option.


\begin{remark}
 To develop with \Peano, you don't have to install it. You can instead just skip
 the installation and work in the download directory.
\end{remark}




\section{Installation test}

Once you have compiled \Peano\, I recommend that you run the unit tests in
directory

\begin{code}
  src/examples/unittests/UnitTests2d
  src/examples/unittests/UnitTests3d
\end{code}

\noindent
You can run these builds with different core counts and also MPI support if you
have compiled with MPI.
The executables contain both node correctness tests and MPI ping-pong tests,
i.e.~you can both assess a valid build plus a working MPI environment.



\section{\Peano\ components and build variants}

\Peano\ currently is delivered as a set of archives, i.e.~static libraries:
\begin{itemize}
  \item There is a technical architecture (\texttt{Tarch}) and the actual
  \texttt{Peano4Core}.
  \item Each archive variant is available as a release version, as a debug
  version, as a version with tracing and assertions and as a tracing-only
  version.
  \item Each archive variant is available as 2d build and as 3d build. If you
  need higher dimensions, you have to build the required libraries manually.
\end{itemize}


The version terminology is as follows:
\begin{itemize}
  \item {\bf debug} The debug versions of the archives all have the postfix
  \texttt{\_debug}. If you link against these versions, the full set of
  assertions, of all tracing and all debug messages is available; though you can
  always filter on the application-level which information you want to see
  (cmp.~Chapter \ref{section:logging}).
  \item {\bf asserts} Thsee versions of the archives all have the postfix
  \texttt{\_asserts}. If you link against these versions, all assertions are on.
  The code also contains tracing macros (see below).
  \item {\bf tracing} The release versions of the archives all have the postfix
  \texttt{\_trace}. If you link against these versions, all assertions and debug
  messages are removed, but some tracing is active. You can switch on/off the
  tracing per class (cmp.~Chapter \ref{section:logging}), and different tracing
  backends allow you to connect to different (profiling) tools.
  \item {\bf release} The release versions of the archives have no
  particular postfix. They disable tracing, any debugging and all assertions.
  These archives should be used by production runs.
\end{itemize}


\noindent
Besides these archives, the \Peano\ installation also comes along with a set of
example applications.
They are found in the directory \texttt{src/examples}.
Most examples create by default two variants of the example: a debug
version and one with only tracing enabled.
Several examples furthermore come along as 2d and 3d build.



\section{Documentation}

\begin{figure}
 \begin{center}
  \includegraphics[width=0.4\textwidth]{10_installation/webpage.png}
  \hspace{0.4cm}
  \includegraphics[width=0.4\textwidth]{10_installation/source-docu.png}
 \end{center}
 \caption{
  \Peano's webpage (left) and a screenshot from the auto-generated source code
  docu which can be reached through the webpage if you don't want to generate
  the pages yourself.
  This documentation also provides indices and a search function as well as all
  documentation formulae typeset with LaTeX.
 }
\end{figure}



There are three major types/resources of documentation of the software:
\begin{enumerate}
  \item This guidebook/cookbook that describes how to use the code base from a
  high abstraction level and with anecdotal examples.
  \item The documentation of the C++ code. Here, I follow an ``everything is in
  the code'' philosophy.
  \item The documentation of the Python code. Here, I follow an ``everything is
  in the code'' philosophy.
\end{enumerate}

\noindent
For the code documentation, ``everything is in the code'' means that all
documentation is comments within the Python script or C++ header files,
respectively.
You can create a webpage from this distributed information through
the tool \texttt{doxygen}\footnote{\url{http://www.doxygen.nl}.}. 


There are two Doxygen configuration files in the repository, i.e.~I keep the
Python and the C++ documentation output separate.
To create the documentation, switch into directory \texttt{src} or
\texttt{python}, respectively. 
In both directories, the Doxygen config file is called
\texttt{peano.doxygen-configuration}, i.e.~calling

\begin{code}
doxygen peano.doxygen-configuration
\end{code}

\noindent
gives you the output. If you prefer not to generate and maintain the
documentation yourself, the \Peano\ webpage hosts the autogenerated
documentation, too.
It is updated roughly once a week.

