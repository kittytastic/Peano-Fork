\chapter{Installation}

There are two ways to obtain Peano: 
You can either download one of the archives we provide on the webpage, or you
can work directly against a repository clone.
If you work with the archive, type in a 
\begin{code}
tar -xzvf myarchive.tar.gz
\end{code}
in the directory where you've stored your downloaded file.


If you work with the git archive, you have to clone this archive first. 
I grant access to Peano free of charge.
However, I ask users to sign up for the software first.
This way I can report to funding agencies how frequent the software is used, and
I also have at least some ideas which application areas benefit from the
software and where it is ran.
\begin{code}
git clone https://gitlab.lrz.de/your-login/Peano.git
cd Peano
git checkout p4
\end{code}


I still maintain the ``old'' Peano in the repository (version 3), and most users
consider this to be the standard Peano generation.
For the present document, it is thus important that you manually switch to the
branch \texttt{p4}.


\section{Configure}
\label{section:installation:configure}


Peano relies on the autotools to set up its build environment.
Change into the project's directory and type in 
\begin{code}
./configure --help
\end{code}


\begin{remark}
 If you clone the repository, i.e.~if you work directly against the git, you
 will have to create the configure scripts manually. 
 This can be achived by running the following six commands:  
 \begin{code}
 libtoolize; aclocal; autoconf; autoheader; 
 cp src/config.h.in .; 
 automake --add-missing
 \end{code}
\end{remark}


The \texttt{--help} option should give you some information about the available
variants of Peano.
In principle, a sole \texttt{./configure} call is sufficient, but you might want
to adopt your build; notably as the default build is serial and does not bring
along any support for postprocessing.
While the help output should be reasonable verbose, I summarise key options
below:

\begin{center}
 \begin{tabular}{lp{10cm}}
  \texttt{--prefix=/mypath} & Will install Peano in \texttt{/mypath}.
   \\
  \texttt{--with-multithreading} & Switch on multithreading. By default, we
  build without multithreading, but a \texttt{--with-multithreading=cpp}, e.g.,
  makes Peano use the C++ threading model. 
   \\
  \texttt{--with-mpi} & Enable the MPI version of Peano. You have to tell the
  build environment however which compile command to use. Please note that
  we need a C++ MPI wrapper. So \texttt{--with-mpi=mpcxx} is a typical call. 
   \\
  \texttt{--with-vtk} & Inform Peano that VTK is available on the system and
  that it should build all the visualisation and postprocessing tools that rely
  upon VTK. You have to pass the directory where the VTK header files are to be
  found, i.e.~you call something alike \texttt{--with-vtk=/usr/include/vtk-8.1}.
   \\
  \texttt{--with-hdf5} & Make Peano support HDF5 output. 
   \\
  \texttt{--with-delta} & Configure Peano such that the geometry library
  $\Delta $ is used. You have to make \texttt{--with-delta=/mydeltasrc} point
  to the $\Delta $ directory.
 \end{tabular}
\end{center}


\section{Build}

Once the configuration has been successful, a simple 
\begin{code}
make
\end{code}
should build the Peano core and some examples.


\begin{code}
make install
\end{code}
finally will deploy the Peano files in the directory specified via the
\texttt{prefix} before.


\section{Peano components and build variants}

Peano currently is delivered as a set of archives, i.e.~static libraries:
\begin{itemize}
  \item There is a technical architecture (\texttt{Tarch}) and the actual
  \texttt{Peano4Core}.
  \item Each archive variant is available as a release version, as a debug
  version, and as a tracing version. 
  \item Each archive variant is available as 2d build and as 3d build. If you
  need higher dimensions, you have to build the required libraries manually.
\end{itemize}


The version terminology is as follows:
\begin{itemize}
  \item {\bf debug} The debug versions of the archives all have the postfix
  \texttt{\_debug}. If you link against these versions, the full set of
  assertions, of all tracing and all debug messages is available; though you can
  always filter on the application-level which information you want to see
  (cmp.~Chapter \ref{section:logging}).
  \item {\bf tracing} The release versions of the archives all have the postfix
  \texttt{\_trace}. If you link against these versions, all assertions and debug
  messages are removed, but some tracing is active. You can switch on/off the
  tracing per class (cmp.~Chapter \ref{section:logging}), and different tracing
  backends allow you to connect to different (profiling) tools.
  \item {\bf release} The release versions of the archives have no
  particular postfix. They disable tracing, any debugging and all assertions.
  These archives should be used by production runs.
\end{itemize}


\noindent
Besides these archives, the Peano installation also comes along with a set of
example applications.
They are found in the directory \texttt{src/examples}.
Most examples create by default two variants of the example: a debug
version and one with only tracing enabled.
Several examples furthermore come along as 2d and 3d build.




% \subsection*{Further reading}
% 
% \begin{itemize}
%   \item Weinzierl, Tobias and Mehl, Miriam (2011). {\em Peano---A Traversal and
%   Storage Scheme for Octree-Like Adaptive Cartesian Multiscale Grids}. SIAM
%   Journal on Scientific Computing 33(5): 2732-2760.
%   \item Bungartz, Hans-Joachim, Eckhardt, Wolfgang, Weinzierl, Tobias and
%   Zenger, Christoph (2010). {\em A Precompiler to Reduce the Memory Footprint of
%   Multiscale PDE Solvers in C++}. Future Generation Computer Systems 26(1): 175-182.
%   \item Bungartz, Hans-Joachim, Mehl, Miriam, Neckel, Tobias and Weinzierl,
%   Tobias (2010). {\em The PDE framework Peano applied to fluid dynamics: an efficient
%   implementation of a parallel multiscale fluid dynamics solver on octree-like adaptive Cartesian grids}. Computational Mechanics 46(1): 103-114.
%   \item   Weinzierl, Tobias (2009). {\em A Framework for Parallel PDE Solvers on
%   Multiscale Adaptive Cartesian Grids}. M\"unchen: Verlag Dr. Hut.
%   \item Bungartz, Hans-Joachim, Mehl, Miriam, Weinzierl, Tobias and Eckhardt,
%   Wolfgang (2008). {\em DaStGen---A Data Structure Generator for Parallel C++
%   HPC Software}. In ICCS 2008: Advancing Science through Computation, Part III.
%   Bubak, van Albada, Sloot and Dongarra, Heidelberg, Berlin: Springer-Verlag.
%   5103: 213-222.
%   \item Brenk, Markus, Bungartz, Hans-Joachim, Mehl, Miriam, Muntean, Ioan
%   Lucian, Neckel, Tobias and Weinzierl, Tobias (2008). {\em Numerical Simulation
%   of Particle Transport in a Drift Ratchet}. SIAM Journal of Scientific Computing 30(6): 2777-2798.
% \end{itemize}
