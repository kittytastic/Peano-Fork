\chapter{Preamble}


\Peano\ is an open source framework for solvers on dynamically adaptive
Cartesian meshes.
Its core is built with C++, but many tools around it are written in Python.
\Peano\  is based upon the fact that spacetrees, a generalisation of the classical octree concept, yield a cascade of adaptive Cartesian grids. Consequently, any spacetree traversal is equivalent to an element-wise traversal of the hierarchy of the adaptive Cartesian grids. The software \Peano\  realises such a grid traversal and storage algorithm, and it provides hook-in points for applications performing per-element, per-vertex, and so forth operations on the grid. It also provides interfaces for dynamic load balancing, sophisticated geometry representations, and other features. Some properties are enlisted below.

Peano is currently available in its fourth generation. 
The development of the original set of Peano codes started around 2002.
2005-2009, we merged these codes into one Peano kernel (2nd generation). 
In 2009, I started a complete reimplementation of the kernel with special
emphasis on reusability, application-independent design and the support for rapid prototyping. 
This third generation of the code ended around 2019 when we released the
ExaHyPE code---a hyperbolic equation system solver engine which uses Peano's
AMR meshes.
The current guidebook discusses \Peano\ , which reuses ideas and lessons learned
from Peano~3 as well as lots of code buildling blocks, but can be seen as a
major rewrite starting from scratch.


\section*{Dependencies and prerequisites}

\Peano's core is plain C++17 code. 
We however use a whole set of tools around it:

\begin{itemize}
  \item GNU autotools (automake) to set up the system (required).
  \item C++17-compatible C++ compiler (required).
  \item Python3 (optional; not required if you work only with the C++
  baseline).
  \item MPI2 (optional). MPI's multithreaded support is required.
  \item Intel's Threading Building Blocks or OpenMP 5. We also can work with C++
  threads, but found their performance suboptimal.
  \item The Visualization Toolkit (VTK) if you want to use the built-in
  visualisation facilities (optional).
  \item Doxygen if you want to create HTML pages of PDFs of the in-code
  documentation.
\end{itemize}

\begin{remark}
I test and maintain \Peano\  for Linux only.
If you prefer Windows or MacOS, it should work as long as you provide the
mandatory tools from above, but I won't be able to help.
\end{remark}


\begin{center}
 \begin{tabular}{|l|l|p{8cm}|}
  \hline
   Software & Version & Remarks \\
  \hline
   GNU g++ & 7.5 or older & Works with \Peano\ core but not with some extensions
   such as \ExaHyPE. \\
   GNU g++ & 9.3.1 & Current development version. Works. \\
  \hline
   Intel icpc & 19.1 & Works with \Peano\ and all
   extensions such as \ExaHyPE. Tests with Intel Fortran have been successful,
   too. \\
  \hline
   VTK & 7--9 & Supported by \Peano. Default is 8, i.e.~if you need other
   version, you have to inform configure (see Chapter \ref{chapter:vtk}). \\
   \hline
   Python & 3 & Development version. Python 2 is not supported by frontend. \\
   \hline
 \end{tabular}
\end{center}

\section*{Who should read this document}

This cookbook is written similar to a tutorial in a hands-on style.
Therefore, it also contains lots of source code snippets.
If you read through a chapter, you should immediately be able to re-program the
presented details in your code and use the ideas.

Therefore, this cookbook is written for people that have a decent programming
background as well as scientific computing knowledge.
% Some background in the particular application area's algorithms for some
% chapters also is required. 
It is a cookbook that shall help readers to translate their scientific vision
into a working code quickly.


\section*{How the text is organised}

% This book covers a variety of problems I have tackled with Peano when I wrote
% scientific papers.
% There is no overall read thread through the document.
% I recommend to start reading some chapters and then jump into chapters
% that are of particular interest.
% Whenever something comes to my mind that should be added, I will add it.
If you feel something is urgently missing and deserves a chapter or things
remain unclear, please write me an email and I'll see whether I can provide some
additional text or extend the cookbook.


{
  \flushright
  \today 
  \\ 
  Tobias Weinzierl 
  \\
}

 