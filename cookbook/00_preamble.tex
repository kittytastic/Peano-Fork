\chapter{Preamble}


 Peano is an open source C++ solver framework. It is based upon the fact that spacetrees, a generalisation of the classical octree concept, yield a cascade of adaptive Cartesian grids. Consequently, any spacetree traversal is equivalent to an element-wise traversal of the hierarchy of the adaptive Cartesian grids. The software Peano realises such a grid traversal and storage algorithm, and it provides hook-in points for applications performing per-element, per-vertex, and so forth operations on the grid. It also provides interfaces for dynamic load balancing, sophisticated geometry representations, and other features. Some properties are enlisted below.

Peano is currently available in its third generation. 
The development of the original set of Peano codes started around 2002.
2005-2009, we merged these codes into one Peano kernel (2nd generation). 
In 2009, I started a complete reimplementation of the kernel with special
emphasis on reusability, application-independent design and the support for rapid prototyping. 
This third generation of the code is subject of the present cookbook.


\section*{Dependencies and prerequisites}

Peano is plain C++ code and depends only on MPI and Intel's TBB or OpenMP if you want to run it with distributed or shared memory support. 
There are no further dependencies or libraries required. 
C++ 11 is used. 
GCC 4.2 and Intel 12 should be sufficient to follow all examples presented in
this document.
If you intend to use Peano, we provide a small Java tool to facilitate rapid
prototyping and to get rid of writing glue code. 
This Peano Development Toolkit (PDT) is pure Java and uses DaStGen. 
While we provide the PDT's sources, there's also a jar file available that comprises all required Java libraries and runs stand alone.
To be able to use DaStGen---we use this tool frequently throughout the
cookbook---you need a recently new Java version.


We recommend to use Peano in combination with
\begin{itemize}
  \item Paraview (\url{www.paraview.org}) or VisIt
  (\url{https://wci.llnl.gov/simulation/computer-codes/visit/}) as our default
  toolboxes create vtk files.
  \item \texttt{make} and \texttt{awk}.
\end{itemize}
But these software tools are not mandatory.


The whole cookbook assumes that you use a Linux system. It all should work on
Windows and Mac as well, but we haven't tested it in detail.


\section*{Who should read this document}

This cookbook is written similar to a tutorial in a hands-on style.
Therefore, it also contains lots of source code snippets.
If you read through a chapter, you should immediately be able to re-program the
presented details in your code and use the ideas.

Therefore, this cookbook is written for people that have a decent programming
background as well as scientific computing knowledge.
Some background in the particular application area's algorithms for some
chapters also is required. 
If you read about the particle handling in Peano, e.g., the text requires you to
know at least some basics such as linked-cell methods.
The text does not discuss mathematical, numerical or algorithmic background.
It is a cookbook after all.


\section*{What is contained in this document}

This book covers a variety of problems I have tackled with Peano when I wrote
scientific papers.
There is no overall read thread through the document.
I recommend to start reading some chapters and then jump into chapters
that are of particular interest.
Whenever something comes to my mind that should be added, I will add it.
If you feel something is urgently missing and deserves a chapter or things
remain unclear, please write me an email and I'll see whether I can provide some
additional text or extend the cookbook.


{
  \flushright
  \today 
  \\ 
  Tobias Weinzierl 
  \\
}

 