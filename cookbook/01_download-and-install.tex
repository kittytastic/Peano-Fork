\chapter{Download and install}

Before you start to use Peano, you need at least two things.

\begin{enumerate}
  \item The Peano source code. Today, the source code consists of two important
  directory. The \texttt{peano} directory holds the actual Peano code. An
  additional texttt{tarch} directory holds Peano's technical architecture.
  \item The Peano Development Toolkit (PDT). The PDT is a small Java archive. It
  takes away the cumbersome work to write lots of glue code, i.e.~empty
  interface implementations, default routines, \ldots, so we use it quite
  frequently.
\end{enumerate}

\noindent
For advanced features, you might want to use some {\bf toolboxes}.
A toolbox in Peano is a small collection of files that you store in a directory
and adopt all pathes accordingly.
From a user's point of view, when we use the term toolbox we actually mean this
directory with all its content.



\begin{history}
Originally, we hoped that Peano's technical architecture (\texttt{tarch}) might
become of value for several projects, i.e.~project appreciate that they do not
have to re-develop things such as logging, writing of output files, writing
support for OpenMP and TBB, and so forth.
To the best of our knowledge, the tarch however is not really used by someone
else, so we cannot really claim that it is independent of Peano.
Nevertheless, we try to keep it separate and not to add anyting AMR or
grid-speicific to the tarch.
\end{history}


\section{Download the archives from the website}

%
% Bilder vom ls
%

\section{Download the Peano sourcecode}


\section{Prepare your own project}
