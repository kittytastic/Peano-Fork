\chapter{Configuring Peano4 in Eclipse}


You can integrate Peano4 straightforwardly into Eclipse. For this, please clone
the repository locally, and invoke

\begin{code}
libtoolize; aclocal; 
autoconf; autoheader; 
cp src/config.h.in .; 
automake --add-missing
\end{code}

\noindent
You might want to check these instructions against the comment in
\texttt{configure.ac}. If you have downloaded a distribution of Peano4,
i.e.~if you have not cloned the raw repository, you can omit these steps.


\section{Integrating the C++ core as project}

This step is required if and only if you want to develop core Peano4 routines.
If you have no intention to do this, you can simple run configure and make on
your git clone and then skip this section. 
Otherwise, open the Eclipse GUI and run the following steps:

\begin{itemize}
  \item Select \texttt{New Project}.
  \item Select a C++ project with the mode \texttt{Managed Build}.
  \item Locate the project at a remote directory (where you git deposits the
  clones).
  \item Select \texttt{GNU Autotools}. The Empty Project is the right one to select.
  \item You need only one target configuration. By default, Eclipse offers you to build Build and Debug. Disable the debug variant.  
\end{itemize}

\noindent
Somee Eclipse wizards tend to overwrite Peano's configure.ac file. If this is the 
case, replace the files through

\begin{code}
git restore  configure.ac
git restore  Makefile.am
git restore  src/Makefile.am
\end{code}

\noindent
and clean the project in Eclipse.


