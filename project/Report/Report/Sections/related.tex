% Exahype templating
% make mroe systems, less user friendly


% Euler 3D SIMD speedup 1.04
% CCZ4 1.27

Much of ExaHyPE's code generation is preformed via templating.
Templating is a powerful, yet simple method to generate large sections of glue code.
Although as discussed in section \ref{sec:problem_statement}, templating along the boundary of user and engine code leaves opportunities for improvements.  
A solution to this problems was proposed by \citeauthor{templateExahype} by extending the domain of templating to user code \cite{templateExahype}.

\citeauthor{templateExahype} observed that while creating an ExaHyPE projects there are often 3 separable roles preformed by users.
These were: application experts, who focus on configuration of ExaHyPE; algorithm experts, who understand and implement the PDEs; and optimisation experts, who work at a lower level to increase performance.
The idea behind further templating is to allow these roles, which may be filled by different people, to operate synergically.
Increased templating also offers a convenient way to implement architecture aware optimisations, increasing code portability and speed.
\citeauthor{templateExahype} showed an example use case of their technique that applied a SIMD friendly SoA to AoS transformation to user code.
This transformation was tested on 2 problems: the Euler equations in 3D, and the Einstein equations from relativistic astrophysics (CCZ4).
They found that the memory bound Euler 3D problem experienced a $1.05\times$ speedup and the compute bound CCZ4 equations experienced a $1.27\times$ speedup.

While templating serves a vital role within ExaHyPE, we do not believe it to be the best way to optimise user code.
Templating inherently requires an optimisation expert for any performance improvements to be realized.
Our compiler based approach is based off the idea that conventional compilers are exceptionally powerful, and they should take the role of optimisation expert, not a person. 



% Yateto used a compiler
% BLAS - faster, need to make tensor contractions
\cite{YATeTo}

% firedrake used a compiler
% Hummm
\cite{FiredrakeAndCOFFEE}


%Auto code gen
% SIMD - problem specific
\cite{codegen_dg_SIMD}

