% Exahype templating
% make mroe systems, less user friendly
\cite{templateExahype}
Another possible method is simply acknowledging that a user will always need to make low-level optimisations and make this process easier.
\citeauthor{templateExahype} introduced such a method for ExaHyPE using the Jinja2 template engine, adding a further level of code generation \cite{templateExahype}.
Using such a method allows for better separation of interests for a user, who in practice is likely a group of people; allowing application experts, algorithm experts and optimisation experts \cite[defined within]{templateExahype} to focus on their specialities.
The template engine also affords optimisation experts more expressive power, allowing them to create general optimisations which are built into system specific optimisations by the template engine. 
Although this method doesn't contribute to the global goal of abstraction it does allow further abstraction within a team. 



% Yateto used a compiler
% BLAS - faster, need to make tensor contractions
\cite{YATeTo}

% firedrake used a compiler
% Hummm
\cite{FiredrakeAndCOFFEE}


%Auto code gen
% SIMD - problem specific
\cite{codegen_dg_SIMD}

