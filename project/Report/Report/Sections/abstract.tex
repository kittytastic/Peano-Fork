ExaHyPE (``An Exascale Hyperbolic PDE Engine'') is a software engine for solving systems of first-order hyperbolic partial differential equations (PDEs).
Hyperbolic PDEs can be used to describe many physical phenomena such as earthquakes, fluid dynamics and gravitational waves.
This paper explores using a compiler based approach to generate fast compute kernels for the Finite Volume scheme, which can be used to solve PDEs within ExaHyPE.
Our compiler \phlat is used to generate Flat Long And potentially Transformed code.
Users describe their problem as a DAG (Directed Acyclic Graph), where every node is a primitive operation, or a DAG itself.
This highly nested structure is then flattened and transformed into code; we target C++.
Generated kernels are typically thousands of lines long.
We found these kernels out preformed the kernels currently used within ExaHyPE by $9\times$ to $16\times$.
We explore why conventional compiles, such as \texttt{gcc}, can better optimise our compiled kernels, deducing that our compiled kernels can be better vectorized on SIMD architectures over hand optimised kernels.
