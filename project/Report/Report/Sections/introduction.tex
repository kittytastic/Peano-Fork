% PDEs - what they are
Many physical phenomena can be described by partial differential equations (PDEs) including:  seismic wave propagation \cite{earthquakePDE}, fluid dynamics \cite{exahype}, and relativistic astrophysics \cite{relativisticPDE}.
As such, PDE solutions can have an important real world impact by, for example, improving tsunami modeling \cite{tsunamiPDE}.


Analytical PDE solutions are challenging to derive, and general solutions for any initial conditions or boundary conditions do not exist.
Instead, we rely on numerical methods to calculate approximate solutions.
However, numerical methods are computationally intensive; increasing the size of the problem, or the accuracy of the solution, increases the computational work required.
Many real world problems require billions of degrees of freedom to model.
The sheer scale of these computations requires the use of high performance computing (HPC), where algorithmic advances and supercomputers make these problems tractable.

In HPC, there are 2 ways to increase computational throughput, either by using more hardware, or by using available hardware more efficiently.
Supercomputing hardware is ever improving, and exa-scale systems are on the horizon.
However, relying on more hardware runs into cost and availability issues.
Therefore, optimising code to better use the hardware is often preferable.
In HPC code there are many levels at which optimisation can occur, one of the most fundamental being improving the throughput of a single core.
There are many example in literature of this having a significant effect on the performance of HPC code \cite{YATeTo,seisolPFLOP,codegen_dg_SIMD}, and this will be the area of optimisation we explore.     

% Exhaype - what it is
Our focus will be on the per core performance of Finite Volume (FV) kernels used to solve PDEs within ExaHyPE (an Exascale Hyperbolic PDE Engine) \cite{exahype}.
Implementing a HPC FV solver is a non-trivial task, and can take research teams months, or even years \cite{tensorChemistry}.
This is the problem that ExaHyPE solves by providing a generalised framework to create solvers for many different problems.
ExaHyPE's domain is linear and non-linear hyperbolic systems of PDEs written in first order form.

The Peano framework \cite{PeanoFramework} lies at the heart of ExaHyPE.
This is responsible for dividing the spacial domain into smaller patches and distributing these patches across computing resources.
Every patch is updated by a single core using the FV scheme.
It is this update process we will optimise.

To use ExaHyPE, a user begins by describing their problem to the ExaHyPE toolkit using a Python interface.
A user will specify information such as: how many unknowns and auxillary variables are used; whether their problem uses a non-conservative product (NCP); what numerical scheme they want to use; if they want to use fixed or adaptive time stepping; how frequently the solution should be plotted; and more.
This is used by the ExaHyPE toolkit to generate a project.
This project is primarily populated by glue code that invokes ExaHyPE, however there are a few placeholder functions that need to be filed in by the user.
These placeholder functions are used to describe the PDE, calculating: flux, eigenvalues, initial conditions, boundary conditions, and non-conservative products (if applicable).
Upon completing these placeholder functions, the project can be compiled and run on systems ranging from a laptop to a supercomputer.

% Exahype - benifits of exahype (fast flexible, friendly)
ExaHyPEs is designed to be fast, flexible and user-friendly; allowing users to create fast programs to solve a wide range of problems while requiring minimal effort from the user.
However, being flexible and user friendly often limits the opportunities for optimistations, especially at the boundary between user and engine code.
In this paper we set out to solve this problem using our compiler based approach \phlat, so called for the Flat Long and potentially Transformed code it produces.
\phlat aims to bridge the gap in ExaHyPE between being flexible and friendly, while producing fast C++.  

\phlat{}'s approach to optimisation is to generate C++ that \texttt{g++}, \texttt{clang++}, \texttt{ipcx}, e.t.c. are more capable of optimising.  
Namely, \phlat can be used to generate monolithic functions that are tens of thousands of lines long.
In this paper we investigate the performance of these kernels and thus how well they are optimised by compilers.

Relying on the compiler for optimisations is often taken for granted, simply appending an \texttt{-O3} flag to the compiler arguments can increase the performance of code by orders of magnitude.
And using a vendor specific compiler, like Intel's \texttt{ipcx}, offers code hardware specific optimisation.
As such, relying primarily on compilers for optimisation is a promising direction for exploration.
Also looking to the future, we see that compilers are only getting more advanced.
Most recently the application of Machine Learning within compilers is being explored \cite{compiler-ml-opt,lots-of-compiler-options}, and development environments such as CompilerGym \cite{compiler-gym} suggest this area of research is set to expand.
Thus relying on compilers may also offer an aspect of longevity to optimisations as well.

% what is in each section
In the next section we go into depth about what problem \phlat solves within ExaHyPE, followed by discussing similar compiler based approaches.
In Section \ref{sec:methodology} we explain the architecture of \phlat and how it is used to generate compute kernels.
In Section \ref{sec:results} we will discuss the performance of kernels generated along with the practicalities of using \phlat.