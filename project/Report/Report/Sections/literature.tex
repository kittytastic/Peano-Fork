% Proj Prep
\textbf{YATeTo:} \cite{YATeTo} describes a tensor toolbox for solving PDEs. 
It motivates using tensors and shows how tensors can be used to describe PDEs. 
It follows on to describe how the toolbox works and the compilation steps from DSL to C++. 
It talks about validating approach on SeisSol vs known benchmark. 
They show using YATeTo is equivalent or faster.

\textbf{ExaHyPE:} \cite{exahype} introduces the ExaHyPE engine. 
Takes you through the concepts used by ExaHyPE to to solve and distribute hyperbolic PDEs. 
Analyse scaling and convergence of ExaHyPE.

\textbf{Templates in ExaHyPE:} \cite{templateExahype} suggests that using ExaHyPE teams are made up of 3 types of people: Application experts with problem specific knowledge; Algorithm experts with knowledge of algorithmic approaches to numerical methods; and Optimisation expert. 
Unlike firedrake which takes a compiler approach this paper looks to use code generation to separate out the roles as much as possible. 
Gives a good overview of how ADER-DG works. 
Presents the idea of using Jninja a html template library to generate the C++ kernels.

\textbf{SeiSol hits PFLOP:} \cite{seisolPFLOP} takes us through the major rework that allows seisol to peek at 1 PFLOP. Importantly using matrix matrix operations in compute kernels.
`` In fully explicit
formulation each of these integration steps is formulated as a compute kernel
that may be expressed as a series of matrix-matrix multiplications''.

\textbf{SeisSol GB Finalist:} \cite{gbseisolPFLOP} is a Gordon Bell finalist.
It is very similar to Seisol hitting 1 PFLOP but optimises for Intel Phi chips.

\textbf{Tensors in Chemistry:} \cite{tensorChemistry} introduces a "program synthesis system" for chemistry where calculations are expressed as tensors.
It has the goal of cutting down development time by introducing a general tool that can generate programs of tensor equations.
There is a lot of emphasis on the size of tensors as they cannot necessarily fit on one nodes memory.
Full of lots of references.

\textbf{Earthquakes are PDEs:} \cite{earthquakePDE} serves 2 purposes. Firstly it is an authoritative source that earthquakes can be modeled by PDEs. Secondly it derives how to use ADER-DG schemes for these equations.

\textbf{Relativistic PDEs:} \cite{relativisticPDE} derives ADER-DG for relativistic effects. Glancing reference.

\textbf{Tsunami PDEs:} \cite{tsunamiPDE} talks about tsunamis and their PDEs. Glancing reference.

\textbf{Peano Framework:} \cite{PeanoFramework} describes the Peano Framework. Glancing reference.

\textbf{UFL for FE:} \cite{UFLforFE} describes a Unified Form Language for finite element methods, subsequently used by firedrake. Glancing reference.

\textbf{Firedrake uses UFL:} \cite{FiredrakeUFL} shows firedrake using UFL. Glancing reference.

\textbf{Roofline model:} \cite{roofline} introduces the roofline model and shows how we get better throughput using SIMD.

\textbf{YATeTo GPU:} \cite{YATeToGPU} trying to get YATeTo on a GPU. 

\textbf{GEMM like tensor contractions:} \cite{GEMMlikeTC} proposes a new method to do tensor contractions. Dissuses and references previous methods Loop-over-GEMM, TT-GEMM-T.


\textbf{Tensors for FV:} \cite{tensorFV}

\textbf{Strength reduction NP-complete:} \cite{strengthReductionNP}

\textbf{COFFEE:} \cite{COFFEE}
\textbf{Cyclops:} \cite{cyclops}

\textbf{TX with Extended BLAS on GPU}: \cite{TCBLASGPU}

% Project