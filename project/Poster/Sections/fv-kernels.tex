A user begins by creating DAGs for the Problem Definition (i.e. flux, ncp, source terms, max eigenvalue functions).
A user then wraps these DAGs in builder functions (A function that when called returns a new version of their DAG).
These builder functions are then passed to a \proc{Patch Update} builder function, which is provided by the engine.
\proc{Patch Update} uses the user's DAGs numerous times, and creates a new user DAG for every single use.

The final \proc{Patch Update} DAG is passed to the FLAT compiler.
Here compiler transforms are preformed, and optional user transforms are preformed.
The final \textbf{C++ output} is typically 1000s to \textbf{10,000s of lines long}, and can be used as a direct replacement in an ExaHyPE project.

