We used 3 test problems: Euler Equations in 2D and 3D (Euler 2D, Euler 3D), and the shallow water equations (SWE).
These 3 problems are provided as examples within the ExaHyPE repository, and those kernels were used as control kernels which we refer to as \textit{default}.
We then recreated these kernels as DAGs and used FLAT to compile them into C++.
Our first method for comparison is a simple synthetic benchmark that measured how many iterations per second a kernel could preform on a fixed input.  

\begin{table}
    \centering
    \begin{tabular}{lllrr}
\toprule
Kernel & Problem & System & Iterations per ms & Speedup \\
\midrule
default & Euler 2D & Intel & 152.90 & - \\
compiled & Euler 2D & Intel & 2486.15 & 16.26 \\
default & Euler 3D & Intel & 50.20 & - \\
compiled & Euler 3D & Intel & 603.29 & 12.02 \\
default & SWE & Intel & 150.98 & - \\
compiled & SWE & Intel & 2313.76 & 15.32 \\
default & Euler 2D & AMD & 363.67 & - \\
compiled & Euler 2D & AMD & 3150.51 & 8.66 \\
default & Euler 3D & AMD & 76.96 & - \\
compiled & Euler 3D & AMD & 663.42 & 8.62 \\
default & SWE & AMD & 252.69 & - \\
compiled & SWE & AMD & 2874.82 & 11.38 \\
\bottomrule
\end{tabular}
 
    \caption{Performance of compiled kernels against default kernel in a synthetic benchmark.} 
\end{table}

Our compiled kernels preformed well, achieving approximately an order of magnitude speedup over the \textit{default} kernels.
This translated to a \textbf{speedup of 5\% - 15\% in ExaHyPE}, which was to be expected for our memory bound test problems.
%\begin{table}
%    \centering
%\begin{tabular}{llrr}
\toprule
Problem & Kernel Type & Run Time (s) & Speedup \\
\midrule
 Euler 2D & default & 64.45 & - \\
 Euler 2D & compiled & 61.38 & 1.05 \\
 Euler 3D & default & 114.55 & - \\
 Euler 3D & compiled & 99.97 & 1.15 \\
 SWE & default & 87.48 & - \\
 SWE & compiled & 78.28 & 1.12 \\
\bottomrule
\end{tabular}
 
%\caption{Performance of compiled kernels against default kernels on the runtime of an ExaHyPE program. Data gathered on the AMD system.}\label{tab:exahype} 
%\end{table}

Our final experiment pitted a hand optimised kernel against a compiled kernel. 
\begin{table}
    \centering
    \begin{tabular}{lrrr}
\toprule
Kernel & Iterations per ms & Speedup vs Default & Speedup vs Handmade \\
\midrule
default & 363.67 & 1.00 & 0.27 \\
handmade & 1364.63 & 3.75 & 1.00 \\
compiled & 3150.51 & 8.66 & 2.31 \\
\bottomrule
\end{tabular}
 
    \caption{Performance of kernel optimised by hand against compiled kernel.}
\end{table}

We found a \textbf{$2.3\times$ speedup of compiled kernels over manual optimisation}.
This was due to \textbf{increased automatic SIMD vectorization} by the \textit{aocc} and \textit{ipcx} compilers.
Using the MAQAO static analysis tool we observed a 60\% floating point vectorization ratio which was almost double that of the hand optimised kernels.