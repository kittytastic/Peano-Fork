We used 3 test problems: Euler Equations in 2D and 3D (Euler 2D, Euler 3D), and the shallow water equations (SWE).
These 3 problems are provided as examples within the ExaHyPE repository and were used as control kernels, referred to as \textit{default}.
We then recreated these kernels as DAGs and used FLAT to compile them into C++.
Our first method for comparison is a simple synthetic benchmark measuring performance of a kernel on a fixed input.  

\begin{table}
    \centering
    \begin{tabular}{lllrr}
\toprule
Problem & Kernel & System & Iterations per ms & Speedup \\
\midrule
Euler 2D & default & Intel & 152.90 & - \\
Euler 2D & compiled & Intel & 2486.15 & 16.26 \\
Euler 3D & default & Intel & 50.20 & - \\
Euler 3D & compiled & Intel & 603.29 & 12.02 \\
SWE & default & Intel & 150.98 & - \\
SWE & compiled & Intel & 2313.76 & 15.32 \\
Euler 2D & compiled & AMD & 3150.51 & 8.66 \\
Euler 3D & compiled & AMD & 663.42 & 8.62 \\
SWE & compiled & AMD & 2874.82 & 11.38 \\
\bottomrule
\end{tabular}
 
    \caption{Performance of compiled kernels against default kernel in a synthetic benchmark.} 
\end{table}

Our compiled kernels preformed well, achieving approximately an order of magnitude speedup over the \textit{default} kernels.
This translated to a \textbf{speedup of 5\% - 15\% in ExaHyPE}, which was to be expected for our memory bound test problems.


\begin{table}
    \centering
    \begin{tabular}{lrrr}
\toprule
Kernel & Iterations per ms & Speedup vs Default & Speedup vs Handmade \\
\midrule
default & 363.67 & 1.00 & 0.27 \\
handmade & 1364.63 & 3.75 & 1.00 \\
compiled & 3150.51 & 8.66 & 2.31 \\
\bottomrule
\end{tabular}
 
    \caption{Performance of kernel optimised by hand against compiled kernel.}
\end{table}
We also compared a hand optimised kernel against a compiled kernel. 
We found a \textbf{$2.3\times$ speedup of compiled kernels over manual optimisation}.
This was due to \textbf{increased automatic SIMD vectorization} by the \textit{aocc} and \textit{ipcx} compilers.
Using the MAQAO static analysis tool \cite{MAQAO}, we observed a 60\% floating point vectorization ratio which was almost double that of the hand optimised kernels.